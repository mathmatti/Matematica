\chapter{Game Theory, boolean functions, etc...}
%% SETS
\newcommand{\playerSet}{ N = \{1, ..., 100\} }
\newcommand{\strategySet}{S_i = \{ 0, 1 \}}
\newcommand{\alternativeSet}{ X = \{x_1, ..., x_m\}}
\newcommand{\strategySets}{S_1 = \{s_{i}^{1}, ..., s_{i}^{|S_i|}\}}
\newcommand{\actionSet}{A_i = \{ a_{i}^{1, ..., a_{i}^{|A_i|}} \}}
\newcommand{\actionProfileSet}{A = A_1 \times ... \times A_n}

%% FUNCTIONS
\newcommand{\setFunction}{p:2^X \to [0,1]}
\newcommand{\utilityFunction}{u_i:\strategies \to R}

%% TUPLES
\newcommand{\strategies}{S_1 \times ... \times S_n}

%\begin{abstract}
%Appunti personali presi durante il corso di Giochi e Modelli Booleani (aka Teoria dei giochi (TG)) - 82114 - ANNO 2017/18 - tenuto dal Prof. Giovanni Rossi. Non avendo seguito tutto il corso, alcuni fatti potrebbero risultare distorti mentre altri potrebbero
%presentare delle lacune.  
%\end{abstract}

%%ciao
%%\chapter{Introduction to Game Theory}
\begin{itemize}
	\item Game theory "begins" in 1944 with the book $[32]$ Games and Economic Behavior by von-Neumann and Morgestern.
	
	\item In 1953 Shapley publishes a fundamental paper $[27]$ defining cooperative games, so that the former ones have been named "non-cooperative" (or strategic) ones thereafter.
	
	\item Given a set $N=\{1,...,n\}$ of $n$ players, a non-cooperative game consists of a product space $S_1 \times ... \times S_n$ of strategies, and $n$ utilities or payoff functions $u_i:S_1 \times ... \times S_n \to R$, $1 \le i \le n$ measuring the "goodness" of strategy profiles $s \in S_1 \times ... \times S_n$ to players $i \in N$. This is the branch of game theory where the famous prisoner's dilemma and Nash equilibrium apply. On the other hand, a cooperative (coalitional) game is a set function $v:2^N \to R_+$ such that $v(0)=0$, where $2^N = \{A:A \subseteq N\}$ is the $2^n$-set of coalitions $A$ or subsets of $N$. Specifically, $v(A)$ is thought of as the worth of cooperation among all (and only) players $i \in A$ (or coalition members).
	\begin{enumerate}
		\item Perch\`e 2?
		\item Ha senso la coalizione in cui sono presenti tutti i players?
	\end{enumerate}
	\item ...
\end{itemize}
%%\chapter{Preferences}
\section{Preferences}
\begin{itemize}
	\item In non-cooperative games (see above), the product space $\strategies = X$ of strategies, over which every player $i \in N$ has preferences in the form of a utility function $\utilityFunction$, is finite (unless otherwise specified).
\end{itemize}

\subsection{Rational preferences}
\begin{itemize}
	\item The primitive ingredient of a choice problem is a set $X$ of alternatives, which may be finite or infinite, and in this latter case either countable or uncountable.
			\begin{enumerate}
				\item L'insieme $X$ di questo punto e the product space $\strategies = X$ sono due cose completamente diverse, giusto?
				\item Qui $X$ \`e un generico insieme e quindi potrebbe anche essere quello del punto precedente? L'importante, come vedremo, che sia definita una relazione di preferenza?
			\end{enumerate}
		
	\item When denoting a finite alternative set by $X = \alternativeSet$, the first $m$ natural numbers $1, ..., m \in N$ are used as distinct "names" for the $|X|=m$ distinct alternatives.
\end{itemize}

\subsection{Utility representation}
%%\chapter{Randomness}
\begin{itemize}
 \item The next step in the study of non-cooperative games is the understanding of strategies. As we shall see, the famous Nash equilibrium surely exists only when players $i \in N$ may each randomize over their finite strategy sets $\strategySets$.
\end{itemize}

\section{Discrete random variables: lotteries}
\begin{itemize}
 \item ...
 \item By the way, also recall that a probability distribution over $X$ is defined as any set function $\setFunction$ satisfying $p(A)+p(B)=p(A \cap B)+p(A \cup B)$ for any two events or subsets (of elementary mutually exclusive events) $A, B \in 2^X$, as well as $p(X)=1$, $p(\o)=0$; then, a main theorem on valuations of distributive lattices (such as Boolean lattice $2^X, \cap, \cup)$) $[1, p. 190]$ entails $p(A) = \sum_{i \in A}p(\{i\})$ for all $A \in 2^X$ (this will be detailed when dealing with the \emph{solution} of coalition (cooperative) games $v:2^N \to R$, see below).
     \begin{enumerate}
      \item Che significa $p(A)+p(B)=p(A \cap B)+p(A \cup B)$?
      \item $[1, p. 190]$ cosa dovrei trovare? non ho capito?
     \end{enumerate}

\section{Probabilities, set functions and voting games}
	\begin{itemize}
		\item Recall that the quantitative notion of probability is associated with \emph{events or subsets $A \in 2^X$ of elementary, mutually exclusive (atomic) events.} In 
	\end{itemize}

\end{itemize}
%%\chapter{Strategies}
\section{Strategies}
\begin{itemize}
	\item In simultaneous-move games all player move or take action simultaneously, hence choosing a strategy is the same as choosing an action. This is no longer true in multistage games, where choosing a strategy means choosing a sequence of (conditional) actions. Although in this course non-cooperative games shall be dealt with only in simultaneous-move form, still multistage games are briefly introduced hereafter to formalize both a general definition of strategies and the notion of (in)complete information.
	\item I - baudo - Multistage games vengono introdotti per generalit\`a e per modellare la nozione di (in)complete information.
	\item before describing multistage games, the simultaneous-move setting enables to distinguish beetween outcomes of the game and strategy/action profiles. To this end, let each player $i \in N$ choose an action from a finite set $\actionSet$, where $|A_i| \ge 2$ for all $i \in N$. The product space $\actionProfileSet$ contains all $n$-tuples or profiles of actions, with generic element $a = (a_1, ..., a_n) \in A$. Preferences  
\end{itemize}


\subsection{Multistage games}
\subsection{Dominated and dominant strategies}
\subsubsection{Strategy deletion}
\subsubsection{Prisoner's dilemma}

\section{Randomization and expected payoffs}
\subsection{Mixed strategies}
\subsection{Domination in mixed strategies}
\subsection{Best responses}
\subsection{Nash equilibrium}
\subsection{Strong equilibrium}
%%\chapter{Esercitazione 1}
Nell'esercizio vedremo alcuni concetti visti durante la prima parte del corso del Prof. Giovanni Rossi.

%\begin{abstract}
%Appunti personali presi durante il corso di Giochi e Modelli Booleani (aka Teoria dei giochi (TG)) - 82114 - ANNO 2017/18 - tenuto dal Prof. Giovanni Rossi. Non avendo seguito tutto il corso, alcuni fatti potrebbero risultare distorti mentre altri potrebbero
%presentare delle lacune.  
%\end{abstract}

%%ciao
\chapter{Definitions}
Cominiciamo col dare le definizioni dei concetti che vengono impiegati nell'esercizio. \\
Per \emph{definizione} intendiamo una formula vera nel linguaggio della matematica basata sugli assiomi della teoria degli insiemi. \\

\section{Pareto}
Pareto cosa? Allora non avendo ben chiaro in mente di cosa andremo a parlare, tuttavia il processo di apprendimento per il tramite della funzione di ripartizione naturale aggrega,
sotto il sostantivo oggettivato \emph{pareto}, alcuni fatti.

Vediamo i fatti separatamente. Utilizzeremo da qui in avanti la p minuscola per indicare che l'entit\`a reale del modello o l'oggetto matematico sottostante possiedono le propriet\`a 
richieste per essere considerate \emph{pareto}.

\subsection{pareto-dominates}


%%\chapter{Types of cooperative games}


	Although in the 70s attention has also been placed on cooperative games with a continuum of players in terms of measure theory (see [5] and related literature), nowadays cooperative games are for the most part dealt with in terms of a finite player set, usually denoted by $N = \{1,...,n\}$. In particular, these games are approached through discrete mathematics as poset/lattice functions. That is, as real-valued functions defined on finite ordered structures. 
	
	%\begin{itemize}
		%\item Continuum of players vuol dire insieme infinito di giocatori oppure insieme finito/infinito di giocatori che pu\`o crescere fino all'infinito? 
		%\item Nel caso moderno, cio\`e attraverso l'utilizzo di insiemi finiti e ordinati, nei ragionamenti l'insieme iniziale dei giocatori rimane fisso oppure può crescere?
		%\item Measure theory? A measure is a generalization of the concepts of length, area, and volume. 
		%\item In [5] what's "value concept"?
		%\item Edgeworthian? 
		%\item Per il momento questo mi basta!!! Il punto introduce quello che servir\`a sapere in seguito: ordered set/lattices e funzioni definite a partire da questi insiemi all'insieme dei numeri reali. Per ulteriori approfondimenti sui giochi with a continuum of players vedi [5] e letteratura affine.
		%\item Since about 1960, attention has focused more and more on games with large masses of players, in which no individual player can affect the overall outcome. Such games arise naturally in the social sciences, as models for situations in which there are large numbers of very "small" individuals, like consumers in an economy or voters in an election. Mathematically, it is often convenient to represent these games with the aid of a "continuum" of players - like the continuum of points on a line or the continuum of drops in a liquid. Represented thus, such games are called non-atomic.[5]
		%\item Quindi i nostri giochi sono atomici?
	%\end{itemize}
		
	Historically, the first cooperative games were defined in 1953 [27] as set functions $v : 2^N \to R_+,v(0)=0$, with subsets $A \in 2^N$ referred to as coalitions (of players). These games may thus be called coalition games, although in many articles and books they are simply named cooperative games, as if exhausting the whole class of cooperative games.
			\begin{enumerate}
				\item Coalition games sono un tipo di cooperative game?
				\item $2^N$ ? qual \`e il significato di questa notazione?
				\item Prospetto?
				\item Essential games?
			\end{enumerate}
	
	Subsequently, in 1963, a further type of cooperative games entered the picture, involving partitions of players or coalition structures, i.e. partitions $P = \{A_1, ..., A_{|P|}\}$ of $N$. In particular, these second-generation cooperative games were named games in partition function form, and they are real-valued functions defined on pairs $(A, P)$ such that $A \in 2^N$ and $P$ is a partition of $N$ such that $A \in P$. These pairs $(A,P)$ are now referred to as "embedded coalitions" (or "embedded subsets" $[13, 14]$). These games pose serious problems in terms of lattice theory, as the corresponding ordered structure (i.e. of embedded subsets) currently needs ad hoc techniques for yielding a lattice (which in any case is not a geometric one, see below). 
			\begin{enumerate}
				\item ...
			\end{enumerate}
		
	Finally, in 1990, a third type of cooperative games was introduced and named "global games" $[12]$. These are simply real-valued partition functions, but still lead to embarassing results when it comes to define and quantify the so-called "solution". Roughly speaking, a solution of a cooperative games should determine the a priori worth, for each player, of playing the game. Somehow overcoming the mainstream literature, in the sequel we shall interpret solutions of cooperative games (of any kind) in terms M\''{o}bius inversion and atomic/geometric lattices.
			\begin{enumerate}
				\item "of any kind" si rifesce ai tre tipi di giochi cooperativi o a tutti i giochi, sia cooperativi che non cooperativi?
			\end{enumerate}	

%%\chapter{Order: posets and lattices}
\section{Order: posets and lattices}
\begin{itemize}
	\item ...
	
	\item Our concern is only with posets or ordered structures which are:
	  \begin{enumerate}
	    \item finite, i.e. $|X|<\infty$,
	    \item with a bottom element $x_{\bot} \in X, i.e. x \ge x_{\bot}$ for all $x \in X$,
	    \item with a top element $x^{\top} \in X$, i.e. $x^{\top} \ge x$ for all $x \in X$.
	  \end{enumerate}
	  In particular, in the sequel our main concern shall be with the poset $(X, \ge)$ given by $(2^N, \supseteq)$ for a finite (player set) $N = \{1,...,n\}$.
	    \begin{enumerate}
	     \item Quindi il top del poset $(2^N, \supseteq)$ is equals to $2^N$ and the bottom of $(2^N, \supseteq)$ is equals to $\O$. 
	    \end{enumerate}
	    
	\item ...
	
	\item For all $x, y \in X$, the corresponding \emph{\bfseries{interval}} (or \emph{\bfseries{segment}} $[25]$) is the subset $[x,y]=\{z:x \le z \le y\} \subseteq X$; hence $x \le y \Rightarrow [x,y] \ne \O$ while $[y,x] = \O$.
	
	\item D - (baudo) - A partially ordered set is \emph{\bf{locally finite}} if each of its intervals has only finitely many elements.
	
	\item A chain is a subset $K \subset X$ any two of whose elements are comparable, i.e. for all $x, y \in K$, either $[x,y] \ne \O$ or else $[y,x] \ne \O$ hold.
	
	\item Dually, an antichain is a subset $AK \subset X$ any two of its elements are uncomparable, i.e. for all $x,y \in AK$, both $[x,y]=\O$ and $[y,x]=\O$ hold.
	
	\item The length of a chain $K = \{x_0,...,x_k\}$ is $|K|-1=k$.

	\item The covering relation, denoted by $>*$, is defined as follows: \\
	      $x>*y \Leftrightarrow [y,x]=\{x,y\}$ (where $\{x,y\} = \{y,x\}$) for all $x,y \in X$.
		  \begin{enumerate}
		   \item Vorrei capire la direzione/terminologia. Anche rispetto al libro!?
		   \item Todo - copiare appunti dal quaderno
		  \end{enumerate}
	\item For $z \ge y$, a $(z-y)$-chain $K_{*}^{z-y} = \{y=x_0, x_1, ..., x_k = z\}$ is said to be maximal if $x_l >* x_{l-1}$ for all $0 < l \le k$.
	    \begin{enumerate}
	     \item from the free dictionary: A sequence of $n+1$ subsets of a set of $n$ elements, such that the first member of the sequence is the empty set and each member of the sequence is a proper subset of the next one.
	    \end{enumerate}

	    
	\item If for any $y, z \in X$ all maximal $(z-y)$-chains have the same length, then poset $(X, \ge)$ is said to satisfy the Jordan-Dedekind JD condition, in which case for every element $x \in X$ the length of any maximal $(x-x_{\bot})$-chain is the \emph{rank} of $x$. Formally, for any poset $(X, \ge)$ with bottom element $x_{\bot}$ and satisfying the JD condition, the rank function $r: X \to Z_+$ is defined recursively by
	\begin{enumerate}
	 \item $r(x_{\bot}) = 0$
	 \item $x >* y \Rightarrow r(x) = r(y)+1$.
	\end{enumerate}
	Thus the rank measures the height of elements (in the Hasse diagram, see above and below).


\end{itemize}

APPUNTI UTILI PER QUESTA SEZIONE
\begin{itemize}
  \item Let $P$ be an ordered set. We say $P$ has a \emph{\bf{top}} element if there exists $\top \in P$ with the property that $x \le \top$ for all $x \in P$.
  \item uniquiness of the top (think of duality), antisymmetry etc.? Il top dovrebbe essere unico perch\`e se supponiamo che esista un altro top $t2$ tale che quindi $x \le t2$ for all $x \in P$ ma allora si avrebbe $\top \le t2 \Rightarrow t2 !\le \top$ per la propriet\`a antisimmetrica pertanto siamo giunti ad una contraddizione perch\`e avevamo supposto $\top$ essere un top. 
  \item Let $P$ be an ordered set and let $S \subseteq P$. An element $x \in P$ is an \emph{\bf{upper bound}} of $S$ if $s \le x$ for all $s \in S$.
  \item Upper bound is unique when it exists.
  \item Quindi posso dire che un top \`e un upper bound che sta dentro l'insieme $P$? Insomma che differenza c\`e tra upper bound e top?
  
  \item In a partially ordered set, an element $p$ emph{covers} an element $q$ when the segment $[q,p]$ contains two elements. $[25, 343]$
  
  \item A lattice is a partially ordered set where max and min of two elements (we call them join and meet, as usual, and write $\vee$ and $\wedge$) are defined. $[25, 342]$
  
  \item A \emph{\bf{segment}} $[x,y]$, for $x$ and $y$ in a partially ordered set $P$, is the set of all elements $z$ between $x$ and $y$, that is, such that $x \le z \le y$. ... . A segment is endowed with the induced order structure; thus, a segment of a lattice is again a lattice. $[25, 342]$
  
  \item A partially ordered set is \emph{\bf{locally finite}} if every segment is finite. $[25, 342]$
  
  \item Let $P$ be a non-empty ordered set. \\
        If $x \vee y$ and $x \wedge y$ exist fora all $x, y \in P$, then $P$ is called a \emph{\bf{lattice}}. \\
        If $\vee S$ and $\wedge S$ exist for all $S \subseteq P$, then $P$ is called a \emph{\bf{complete lattice}}.
        
  \item Totally ordered subsets of a poset play an important role in the theory of partial orders.
  
  \item 
\end{itemize}

\subsection{Maximal chains of subsets and permutations}
\subsection{Subset of Boolean lattices}
\subsubsection{Atomicity}
\subsubsection{Complementation}
%%\chapter{M\"obius inversion}

	M\"obius inversion applies to any (locally finite) poset, provided a bottom element exists [25]. For the Boolean lattice $(2^N, \cap, \cup)$ of subsets of $N$ ordered by inclusion $\supseteq$ and the geometric lattice $(2^N, \vee, \wedge)$ of partitions of $N$ ordered by coarsening $\ge$ $[1, 31]$, the bottom elements are, respectively, the empty set $\O$ and the fines partition $P_\bot = {{1}, ..., {n}}$.
		\begin{enumerate}
			\item Locally finite poset - where all intervals are finite
			\item Boolean lattice - 
			\item Geometric lattice
		\end{enumerate} 

%%\chapter{Incidence algebra}
\subsection{Incidence algebra}
\subsection{M\"obius inversion}
\subsection{Vector spaces and bases}
\subsection{Lattice functions}
\subsubsection{Set functions and Boolean or Pseudo-Boolean functions}
\subsubsection{Polynomial multilinear extension of set functions}

\section{Formulario di Teoria dei Giochi}


%% LANGUAGE
$A, B, X, R, N$, etc. for SETS
$a, b, x, r, n$, etc. for ELEMENTS of above sets
$i, j, n, x, y, z$, etc for INDICES of SETS or ELEMENTS
$()$, for function application and tuples
Indices are use to iterate over a set or to name the object to which belongs.

Doesn't exist a way to simulate hash map in mathematics. 

Dobbiamo inventare un operatore o una struttura dati in grado di rappresentare in modo funzionale ai calcoli un hash map ovvero un oggetto che possiede, si porta dientro con se a sua volta un insieme.

Vi \`e quasi una certa ridondanza in questo fatto.

Ma in realt\`a si tratta solo di comprendere il significato dell'applicazione/funzione/mappa che assegna ad ogni elemento di un insieme, un altro insieme.


\begin{verbatim}
	%% ATTENZIONE
	l'indice n a volte seve per dare un nome e un numero, altre volte solo per dare un numero, 
	per esempio nella formula d_1, ..., d_n l'indice n significa prendo enne indici
	senza che faccia specificatamente ad n che rappresenta il numero dei giocatori. 
	n indica due cose differenti!!! non sempre quando trovate n vuol dire che tale n si
	intenda il numero di giocatori 

	%% VALUE
	Si definisce value un qualunque numero reale. 

    %% VOGLIAMO QUANTIFICARE (VALORIZZARE) - valori di R
    LET R be REAL SET
    LET v in R
    
    %% VOGLIAMO CONTARE (COUNT)            - valori di N 
    LET n in N
    LET A be SET
    LET |A| := n 
    
    
    %% VOGLIAMO RANDOMIZZARE. In questo caso occorre il value di un intero insieme.
    LET D be SET
    LET n INDICE of N
    LET D = {d_1, ..., d_n}   
    LET [0,1] \in R
    LET d_1, ..., d_n \in [0,1]
    LET d_1 + ... + d_n = 1    
    Ora che abbiamo costruito D (o DELTA) ovvero l'insieme randomizzatore 
    lo possiamo utilizzare nei nostri calcoli.
    
    
	LET A, B be SETS
	LET B = {b}
	LET b in R
	LET {B} be an anonymous SET. In this case, {B} is a set with two elements: 
	B (which is a set) and 0 (the empty set). 
	LET a elements of A 
	LET value be a FUNCTION
	LET value A \to {B} -- ABSTRACTION
	LET value defined as a --> B in words: each elements of A is mapped to the same set B. 
	    This fact is useful for successive steps, counting how many of...	    	    
	LET value(a) --> SUM of elements of B
\end{verbatim}

Now we are able to define randomness

\begin{verbatim}
	LET A, B_1, ... B_n , {B_1, ..., B_n} be SETS   %% Here n is just an INDICES
	LET a elements of A
	LET value be a FUNCTION
	LET value A \to {B_1, ..., B_n} -- ABSTRACTION
	LET value defined as 
\end{verbatim}



So we are conducted to the definition of value of an element of a set. Whenever I get an element of a set I can ask for its value.

In questo modo non basta pi\`u dire che $A$ e $B$ sono insiemi ma occorre anche specificare la funzione value che restituisce il valore di un suo elemento, cioè una funzione che dato in input un elemento dell'insieme restituisce un valore.

Per convenzione e per semplificare i calcoli assumiamo che le funzioni value restituiscano sempre un insieme e che il value è dato dalla somma dei valori presenti nell'insieme restituito.

Let's define \emph{Pareto}

Prima per\`o proviamo a fare il passaggio successivo, prima abbiamo imparata a calcolare il valore del payoff di un giocatore ma questa cosa pu\`o anche essere sbagliata dal punto di vista
della teoria dei giochi cooperativi. Adesso vogliamo introdurre la nozione di payoff per coalizione che come si pu\`o immaginare \`e la somma dei payoff dei singoli giocatori.

Altra cosa importante sar\`a quello di attivare il meccanismo della partizione (partizionamento) e della conseguente enumerazione di sotto-insiemi che godono di determinate propriet\`a.

Ed infine occorrer\`a passare alle boolean function.






%%%%%%%%%%%%%% DA SISTEMARE
%\section{Solutions of coalitional games}
%\subsection{Axiomatic characterization of solutions}
%\subsection{The Shapley value: existence and uniqueness}
%\subsubsection{Integrating MLE first derivatives}
%\subsection{Random-order and probabilistic solutions}
%\subsection{Weights and Shapley values}
%\subsection{Core of coalitional games}
%\subsection{Cooperation restrictions}
%\subsubsection{Partition constraints type I}
%\subsubsection{Partition constraints type II}
%\subsubsection{Graph-restricted games}

%\section{Geometric lattices}
%\subsection{Two examples by means of partitions}
%\subsection{Cooperation restrictions in general}

%\section{Coalition formation}
%\subsection{Network formation}
%\subsection{Fuzzy coalitions}

%\section{Set packaging and partitioning in combinatorial optimization}




