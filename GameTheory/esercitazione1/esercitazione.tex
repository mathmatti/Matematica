%\documentclass[]{article}
\documentclass[a4paper,12pt]{book}
\usepackage[utf8]{inputenc}
\usepackage{graphicx}

\usepackage{color}
\usepackage{amsmath}



%% SETS - ABSTRACTIONS - OBJECTS
\newcommand{\playerSet}{ N = \{1, ..., 100\} }
\newcommand{\strategySet}{S_i = \{ 0, 1 \}}
\newcommand{\alternativeSet}{ X = \{x_1, ..., x_m\}}
\newcommand{\strategySets}{S_1 = \{s_{i}^{1}, ..., s_{i}^{|S_i|}\}}
\newcommand{\actionSet}{A_i = \{ a_{i}^{1, ..., a_{i}^{|A_i|}} \}}
\newcommand{\actionProfileSet}{A = A_1 \times ... \times A_n}

%% FUNCTIONS
\newcommand{\setFunction}{p:2^X \to [0,1]}
\newcommand{\utilityFunction}{u_i:\strategies \to R}

%% TUPLES - ELEMENTS - OBJECTS
\newcommand{\strategies}{S_1 \times ... \times S_n}
\newcommand{\x}{×}






\begin{document}

\author{giuseppe baudo}
\title{Esercizi svolti di Teoria dei Giochi}
\date{October 2017}


\frontmatter
\maketitle
\tableofcontents

%\begin{abstract}
%Appunti personali presi durante il corso di Giochi e Modelli Booleani (aka Teoria dei giochi (TG)) - 82114 - ANNO 2017/18 - tenuto dal Prof. Giovanni Rossi. Non avendo seguito tutto il corso, alcuni fatti potrebbero risultare distorti mentre altri potrebbero
%presentare delle lacune.  
%\end{abstract}

%%ciao
\chapter{Definitions}
Cominiciamo col dare le definizioni dei concetti che vengono impiegati nell'esercizio. \\
Per \emph{definizione} intendiamo una formula vera nel linguaggio della matematica basata sugli assiomi della teoria degli insiemi. \\

\section{Pareto}
Pareto cosa? Allora non avendo ben chiaro in mente di cosa andremo a parlare, tuttavia il processo di apprendimento per il tramite della funzione di ripartizione naturale aggrega,
sotto il sostantivo oggettivato \emph{pareto}, alcuni fatti.

Vediamo i fatti separatamente. Utilizzeremo da qui in avanti la p minuscola per indicare che l'entit\`a reale del modello o l'oggetto matematico sottostante possiedono le propriet\`a 
richieste per essere considerate \emph{pareto}.

\subsection{pareto-dominates}


\chapter{Esercizio}




%%%%%%%%%%%%%% DA SISTEMARE
%\section{Solutions of coalitional games}
%\subsection{Axiomatic characterization of solutions}
%\subsection{The Shapley value: existence and uniqueness}
%\subsubsection{Integrating MLE first derivatives}
%\subsection{Random-order and probabilistic solutions}
%\subsection{Weights and Shapley values}
%\subsection{Core of coalitional games}
%\subsection{Cooperation restrictions}
%\subsubsection{Partition constraints type I}
%\subsubsection{Partition constraints type II}
%\subsubsection{Graph-restricted games}

%\section{Geometric lattices}
%\subsection{Two examples by means of partitions}
%\subsection{Cooperation restrictions in general}

%\section{Coalition formation}
%\subsection{Network formation}
%\subsection{Fuzzy coalitions}

%\section{Set packaging and partitioning in combinatorial optimization}


\backmatter
% bibliography, glossary and index would go here.

\end{document}
