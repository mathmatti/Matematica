\chapter{Preferences}
\section{Preferences}
\begin{itemize}
	\item In non-cooperative games (see above), the product space $\strategies = X$ of strategies, over which every player $i \in N$ has preferences in the form of a utility function $\utilityFunction$, is finite (unless otherwise specified).
\end{itemize}

\subsection{Rational preferences}
\begin{itemize}
	\item The primitive ingredient of a choice problem is a set $X$ of alternatives, which may be finite or infinite, and in this latter case either countable or uncountable.
			\begin{enumerate}
				\item L'insieme $X$ di questo punto e the product space $\strategies = X$ sono due cose completamente diverse, giusto?
				\item Qui $X$ \`e un generico insieme e quindi potrebbe anche essere quello del punto precedente? L'importante, come vedremo, che sia definita una relazione di preferenza?
			\end{enumerate}
		
	\item When denoting a finite alternative set by $X = \alternativeSet$, the first $m$ natural numbers $1, ..., m \in N$ are used as distinct "names" for the $|X|=m$ distinct alternatives.
\end{itemize}

\subsection{Utility representation}