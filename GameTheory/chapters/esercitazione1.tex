\chapter{Esercitazione 1}
Nell'esercizio vedremo alcuni concetti visti durante la prima parte del corso del Prof. Giovanni Rossi.

%\begin{abstract}
%Appunti personali presi durante il corso di Giochi e Modelli Booleani (aka Teoria dei giochi (TG)) - 82114 - ANNO 2017/18 - tenuto dal Prof. Giovanni Rossi. Non avendo seguito tutto il corso, alcuni fatti potrebbero risultare distorti mentre altri potrebbero
%presentare delle lacune.  
%\end{abstract}

%%ciao
\chapter{Definitions}
Cominiciamo col dare le definizioni dei concetti che vengono impiegati nell'esercizio. \\
Per \emph{definizione} intendiamo una formula vera nel linguaggio della matematica basata sugli assiomi della teoria degli insiemi. \\

\section{Pareto}
Pareto cosa? Allora non avendo ben chiaro in mente di cosa andremo a parlare, tuttavia il processo di apprendimento per il tramite della funzione di ripartizione naturale aggrega,
sotto il sostantivo oggettivato \emph{pareto}, alcuni fatti.

Vediamo i fatti separatamente. Utilizzeremo da qui in avanti la p minuscola per indicare che l'entit\`a reale del modello o l'oggetto matematico sottostante possiedono le propriet\`a 
richieste per essere considerate \emph{pareto}.

\subsection{pareto-dominates}

