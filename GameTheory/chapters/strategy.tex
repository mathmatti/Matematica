\chapter{Strategy}
Che cos'\`e una strategia? Valuta i prospetti. Tutti i prospetti. Supponi per un attimo di essere un essere superiore e di conoscere tutti i possibili risultati dell'interazione tra due entit\`a. Ops, scusate stavo correndo troppo. Supponiamo che le entit\`a siano invece due persone, la persona (o player) $p_1$ e la persona $p_2$.
Bene che cosa puoi fare? Bh\`e puoi pensare per esempio questo: "Se solo i giocatori conoscessero tutti i possibili esiti del gioco, di sicuro farebbero la scelta/mossa/strategia giusta". Ed infatti, per certi aspetti, nella teoria dei giochi \`e proprio cos\`i. Si dice con linguaggio tecnico che i players conoscono tutti i possibili prospetti(da prospetto: guardare innazi) del gioco. Se i giocatori hanno la possibilit\`a di guardare tutti i prospetti (anche se randomizzati) allora si dice che il gioco \`e ad \emph{informazione perfetta}, altrimenti, se alcuni prospetti non sono noti per qualche giocatore, allora il gioco si dice ad \emph{informazione incompleta}.  

Prima di addentrarci in formalismi che riguardano insiemi, ennuple, sottoinsiemi e mappe, lasciamo la parola a chi ha iniziato la discipline della teoria dei giochi.

\section{Explanation of the Termini Technici. \cite{vonNeumann1944}}
Chapter II, GENERAL FORMAL DESCRIPTION OF GAMES OF STRATEGY, 6 The Simplified Concept of a Game, 6.1 Explanation of the Termini Technici.

Before an exact definition of the combinatorial concept of a game can be given, we must first clarify the use of some termini. There are some notions which are quite fundamental for the discussion of games, but the use of which in everyday language is highly ambiguous. The words which describe them are used sometimes in one sense, sometimes in another, and occasionally - worst of all - as if they were synonyms. We must therefore introduce a definite usage of \emph{termini technici}, and rigidly adhere to it in all that follows.

First, one must distinguish between the abstract concept of a \emph{game}, and the individual \emph{plays} of that game. The \emph{game} is simply the totality of the rules which describe it.

\begin{definizione}
 A \emph{game} is the totality of the rules which describe it.
\end{definizione}

Every particular instance at which the game is played - in a particular way - from beginning to end, is a \emph{play}. In most games everday usage calls a play equally a game; thus in chess, in poker, in many sports, etc. In Bridge a play corresponds to a "rubber" in Tennis to a "set" but unluckily in these games certain components of the play are again called "games". The French terminology is tolerably unambiguous: "game" = "jeu", "play" = "partie".

Second, the corresponding distinction should be made for the moves, which are the component elements of the game. A move is the occasion of a choice between various alternatives, to be made either by one of the players, or by some device subject to chance, under conditions precisely prescribed by the rules of the game. The \emph{move} is nothing but this abstract "occasion", with the attendant details of description, - i.e. a component of the "game". The specific alternative chosen in a concrete instance - i.e. in a concete \emph{play} - is the \emph{choice}. Thus the moves are related to the choices in the same way as the game is to the play. The game consists of a sequence of moves, and the play of a sequence of choices. In this sense we would talk in chess of the first move, and of the choice "E2-E4".

Finally, the \emph{rules} of the game should not be confused with the \emph{strategies} of the players. Exact definitions will be given subsequently, but the distinction which we stress must be clear from the start. Each player selects his strategy - i.e. the general principles governing his choices - freely. 
While any particular strategy may be good or bad - provided that these concepts can be interpreted in an exact sense (cf. 14.5. and 17.8-17.10.) - it is within the player's discretion to use or to reject it. The rules of the game, however, are absolute commands. If the are ever infringed, then whole transaction by definition ceases to be game described by those rules. In many cases it is even physically impossible to violate them. E.g.: In Chess the rules of the game forbid a player to move his king into a position of "check". This is a prohibition in the same absolute sense in which he may not move a pawn sideways. But to move the king into a position where the opponent can "checkmate" him at the next move is merely unwise, but not forbidden.

