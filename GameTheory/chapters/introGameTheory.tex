\chapter{Introduction to Game Theory}
\begin{itemize}
	\item Game theory "begins" in 1944 with the book $[32]$ Games and Economic Behavior by von-Neumann and Morgestern.
	
	\item In 1953 Shapley publishes a fundamental paper $[27]$ defining cooperative games, so that the former ones have been named "non-cooperative" (or strategic) ones thereafter.
	
	\item Given a set $N=\{1,...,n\}$ of $n$ players, a non-cooperative game consists of a product space $S_1 \times ... \times S_n$ of strategies, and $n$ utilities or payoff functions $u_i:S_1 \times ... \times S_n \to R$, $1 \le i \le n$ measuring the "goodness" of strategy profiles $s \in S_1 \times ... \times S_n$ to players $i \in N$. This is the branch of game theory where the famous prisoner's dilemma and Nash equilibrium apply. On the other hand, a cooperative (coalitional) game is a set function $v:2^N \to R_+$ such that $v(0)=0$, where $2^N = \{A:A \subseteq N\}$ is the $2^n$-set of coalitions $A$ or subsets of $N$. Specifically, $v(A)$ is thought of as the worth of cooperation among all (and only) players $i \in A$ (or coalition members).
	\begin{enumerate}
		\item Perch\`e 2?
		\item Ha senso la coalizione in cui sono presenti tutti i players?
	\end{enumerate}
	\item ...
\end{itemize}