\chapter{Games classification}
Come sono classificati i giochi? Cio\`e qual'\`e la terminologia adottata se facciamo variare alcune variabili/propriet\`a dei termini/oggetti coinvolti ossia players, randomizzazione, payoffs, alternative.

\section{General Principles of Classification and of Procedure. \cite{vonNeumann1944}}
Chapter II, GENERAL FORMAL DESCRIPTION OF GAMES OF STRATEGY, 5 Introduction, 5.2 General Principles of Classification and of Procedure.

[5.2.1.] Certain aspects of "games of strategy" which were already prominent in the last sections of Chapter I will not appear in the beginning stages of the discussions which we are now undertaking. Specifically: There will be at first no mention of coalitions between players and the compensations which they pay to each other. (Concerning these, cf. 4.3.2., 4.3.3., in Chapter I).

\quote{Per cui all'inizio trascuriamo il fatto che i giocatori possano cmq aiutarsi gli uni e gli altri. Da questo pu\`o nascere una prima confusione, ossia il fatto che si possa pensare che la teoria dei giochi sia suddivisa in teoria dei giochi cooperativi e non. come principale criterio di suddisione. In realt\`a nel testo \cite{vonNeumann1944} la prima distinzione fondamentale \`e tra giochi a somma zero e giochi a somma diversa da zero, diciamo positiva. Ma cosa devo sommare?}

We give a brief account of the reasons, which will also throw some light on our general disposition of the subject.

An important viewpoint in classifying games is this: Is the sum of all payments received by all players (at the end of the game) always zero; or is this not the case? If it is zero, then one can say that the players pay only to each other, and that no production or destruction of goods is involved. All games which are actually played for enternainment are of this type. But the economically significant schemes are most essentially not such. There the sum of all payments, the total social product, will in general not be zero, and not even constant. I.e., it will depend on the behavior of the players - the participants in the social economy. This distinction was already mentioned in 4.2.1., particularly in footnote 2, p.34. We shall call games of the first-mentioned type \emph{zero-sum} games, and those of the latter type \emph{non-zero-sum} games.

We shall primarily construct a theory of the zero-sum games, but it will be found possible to dispose, with its help, of all games, without restriction. Precisely: We shall show that the general (hence in particular the variable sum) $n$-person game can be reduced to a zero-sum $n+1$-person game. (Cf. 56.2.2.) 

\quote{Wow calma un attimo. Di cosa stiamo parlando? Induzione matematica? cio\`e da dove saltano fuori $n$ ed $n+1$?}

Now the theory of the zero-sum $n$-person game will be based on the special case of the zero-sum two-person game. (Cf. 25.2). Hence our discussion will begin with a theory of these games, which will indeed be carried out in Chapter III.

Now in zero-sum two person games coalitions and compensantions can play no role. 

\quote{The only fully satisfactory "proof" of this assertion lies in the construction of a complete theory of all zero-sum two-person games, whithout use of those devices. This will be done in Chapter III, the decisive result being contained in 17. It ought to be clear by common sense, howerver, that "understandings" and "coalitions" can have no role here: Any such arrangement must involve at least two players - hence in this case all players - for whom the sum of payments is identically zero. I.e. there are no opponents left and no possible objectives.}

The questions which are essential in these games are of a different nature. These are the main problems: How does each player plan his course - i.e. how does one formulate an exact concept of a strategy? What information is available to each player at every stage of the game? What is the role of a player being informed about the other player's strategy? About the entire theory of the game?