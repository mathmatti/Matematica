\section{Teoria dei Gruppi, Group Theory}
\begin{definizione}
Si chiama \textit{Gruppo} un qualunque insieme non vuoto $G$ in cui è possibile definire un'operazione da $G$ in $G$ che abbia le seguenti caratteristiche:
\begin{itemize}
 \item $a, b \in G$ implies that $a*b \in G$. (We describe this by saying that $G$ is \textit{closed} under $*$).
 \item Given $a, b, c \in G$, then $a * (b*c) = (a*b)*c$. (This is described by saying that the \textit{associative law} holds in $G$).
 \item There exists a special element $e \in G$ such that $a*e=e*a=a$ for all $a \in G$. ($e$ is called the \textit{identity} or \textit{unit element} of $G$).
 \item For every $a \in G$ there exists an element $b \in G$ such that $a*b=b*a=e$. (We write this element $b$ as $a^{-1}$ and call it the \textit{inverse} of $a$ in $G$).
\end{itemize}
\end{definizione}

\begin{osservazione}
L'insieme $A(S)$ di tutte le permutazioni con l'operazione di composizione tra funzioni.
\end{osservazione}

\begin{definizione}
A group $G$ is said to be a \textit{finite group} if it has a finite number of elements.
\end{definizione}

\begin{definizione}
The order of a group is its cardinality, i.e., the number of elements in its set. Also, the order, sometimes period, of an element
$a$ of a group is the smallest positive integer $m$ such that $a^m=e$ (where $e$ denotes the identity element of the group, and $a^m$
denotes the product of $m$ copies of $a$). If no such $m$ exists, $a$ is said to have infinite order.
\end{definizione}

\begin{osservazione}
NB: La stessa definizione può essere data con la notazione additiva.
\end{osservazione}

\begin{osservazione}
The order of a group $G$ is denoted by $ord(G)$ or $|G|$ and the order of an element $a$ is denoted by $ord(a)$ or $|a|$. 
\end{osservazione}

\begin{definizione}
A nonempty subset, $H$, of a group $G$ is called a $subgroup$ of $G$ if, relative to the product in $G$, $H$ itself forms a group.
\end{definizione}

\begin{osservazione}
We stress the phrase "relative to the product in $G$". Take, for instance, the subset $A = \{1, -1\}$ in $Z$, the set of integers. Under the multiplication of integers,
$A$ is a group. But $A$ is not a subgroup of $Z$ viewed as a group with respect to $+$.
\end{osservazione}

\begin{osservazione}
From intro to paragraph about Subgroup From Abstract Algebra by Herstein

In order for us to find out more about the makeup of a given group $G$, it may be too much of a task to tackle all of $G$ head-on. It might be desiderable to focus our
attention on appropriate pieces of $G$, which are smaller, over which we have some control, and are such that the information gathered about them can be used ot get 
relevant information and insight about $G$ itself. The question then becomes: What should serve as suitable pieces for this kind of dissection of $G$? Clearly, whatever
we choose as such pieces, we want them to reflect the fact that $G$ is a group, not merely any old set.

A group is distinguished from an ordinary set by the fact that it is endowed with a well-behaved operation. It is thus natural to demand that such pieces above behaved
reasonably with respect to the operation of $G$. Once this is granted, we are led almost immediately to the concept of a subgroup of a group.
\end{osservazione}




\section{DEFINIZIONE}
A cyclic group or monogenous (ermafrodita?, monogenetico?) group is a group that is generated by a single element. Thas is, it consist
of a set of elements with a single invertible associative operation, and it contains an element $g$ such that every other element
of the group may be obtained by repeatedly applying the group operation or its inverse to $g$. Each element can be written as a power
of $g$ in multiplicative notation, or as a multiple of $g$ in additive notation. this element $g$ is called a generator of the group. \cite{cyclicgroup1}

\section{APPROFONDIMENTI}
\begin{itemize}
 \item EN.WIKIPEDIA.ORG: Cyclic group \cite{cyclicgroup1}
 \item EN.WIKIPEDIA.ORG: Generating set of a group \cite{cyclicgroup2}
 \item DISPENSA: Elementi periodici. Teoremi di Lagrange, Eulero e Fermat. Gruppi ciclici. \cite{cyclicgroup3}
\end{itemize}




\section{Definizione}
Sia $f:G \longrightarrow G'$ un omomorfismo di gruppi.
Chiamiamo nucleo di $f$, e lo indichiamo con $ker f$ (oppure $ker(f)$), il sottoinsieme $f^{-1}({1'})$, dove $1'$ è l'elemento netro di $G'$. \cite{progmat1}

In questa definizione occorre notare che per trovare il $ker$ bisogna trovare l'insieme degli elementi neutri di $G'$ e tramite l'inversa risalire agli elementi del ker.

\section{Prerequisites}
\begin{itemize}
 \item Elemento neutro
\end{itemize}




\section{Definizione}
Dati due gruppi $(G,\cdot)$ e $(H, \circ)$, una funzione $f:G \longrightarrow H$ è un omomorfismo se
$f(a \cdot b) = f(a) \circ f(b)$


\section{Approfondimenti}
capitolo del libro ARTIN



\section{ENUNCIATO}

\section{DIMOSTRAZIONE}

\section{APPROFONDIMENTI}
\begin{itemize}
 \item EN.WIKIPEDIA.ORG: Lagrange's theorem (group theory) \cite{orderofgroup2}
\end{itemize}


