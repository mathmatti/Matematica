\section{Introduzione, Introduction}		
Mathematics è il tentativo di raccolta di appunti e materiale per studiare con profitto nei corsi di laurea in Matematica.

Il programma di riferimento \`{e}, principalmente, quello dell'Universit\`{a}
di Bologna, anche se, non essendo iscritto a nessuno corso, ho cercato di integrare i programmi (syllabus) con i corsi di altre Universit\'{a}. Diciamo che la matematica \'{e} tale.
Comunque, nel confrontare i vari corsi proposti dalle varie universit\'{a} mi sono accorto che argomenti trattati in geometria vengano trattati in altri corsi denominati algebra lineare oppure, geometria e algebra.

Pertanto, mi sono preso la libert\'{a} di organizzare tutti i concetti di tutte le materie in un unico corpo. Questo punto non \`{e} banale perch\'{e} apre alcune questioni oggi
approfondite in corsi quali \textit{Interazione persona-computer}, \textit{Semantic web}, \textit{Intelligenza artificiale}, \textit{Logica}, \textit{Machine learning}, \textit{Mathematical Knowledge Management}, etc.

Il materiale raccolto \'{e} ancora in fase embrionale. 


\section*{Nota}
Il materiale contenuto nella cartella di google drive è ad accesso limitato. Per accedere cliccare sul link seguente e seguire le istruzioni per ottenere
l'accesso alla cartella. \url{https://drive.google.com/drive/folders/0Bx2fZ0r5vhSSSDdvWkVjNG9YQjQ}

\section*{Introduzione}
I libri che trovate nella \href{https://drive.google.com/drive/folders/0Bx2fZ0r5vhSSSDdvWkVjNG9YQjQ}{folder}, sono stati raccolti seguendo la bibliografia proposta dai Docenti di Università
italiane e straniere. Si trovano i classici di algebra, analisi, geometria, etc. Inoltre, ho selezionato alcuni libri perchè hanno una data stampa risalenete al massimo
agli ultimi tre anni che in genere sono fatti bene perchè raccolgono le esperienze maturate studiando i testi che li hanno preceduti.

Oltre a libri, troverete dispense, papers, etc. Al momento non ho fatto distinzione tra libri o altro pdf ma ho semplicemente suddiviso il materiale seguendo
più o meno le materie indicate nel \href{Syllabus.html}{Syllabus} e quindi troverete le seguenti cartelle: Algebra, Analisi, Topologia, Geometria, etc.

\section*{Come utilizzare il materiale pdf}
Esistono buoni articoli che ci danno una panoramica su come utilizzare un libro di testo o una dispensa. Alcuni professori, specie nelle lezioni introduttive, danno
informazioni riguardanti lo studio della materia nel suo complesso e con esso anche un accenno sull'utilizzo dei libri di testo.

