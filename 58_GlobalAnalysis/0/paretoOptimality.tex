\chapter{Pareto Optimality}

\section{Pareto Optimality and Social Optimality. \cite{kleinberg2010}}

In a Nash equilibrium, each player's strategy is a best response to the other player's strategy....???

Pareto Optimality. The first definition is 

\section{Pareto Optimality. \cite{bauso2014}}
We conclude this chapter (chap.4) introducing a property which may or may not be enjoyed by equilibria, that is, Pareto optimality. 
It must be said that, equilibria that are also Pareto optimal represent extremely stable solutions in that not only no player is better off by changing actions, but also no players can be better off by jointly deviating without causing a loss for at least one player.

\quote{La cosa interessante is that pareto-optimality is defined as a property, e non importa se questa propriet\`a \`e in qualche modo connessa al concetto di equilibrio. Infine si ricorda che un equilibrio che sia anche pareto ottimale rappresenta l'eccezione e non la regola, semmai ci fosse una conessione di dipendenza tra il concetto di equilibrio e quello di paretto ottimalit\`a}

\begin{definizione}
	A pair of strategies $(a_{1}^{PO}, a_{2}^{PO})$ is said to be Pareto optimal (PO) if there exists no other pair $(a_1,a_2)$ such that for $i=1,2$
		\[
			u_i(a_1, a_2) > u_i(a_{1}^{PO}, a_{2}^{PO}) \wedge u_{-i} \ge u(a_{1}^{PO}, a_{2}^{PO}) 
		\]
\end{definizione}	

Ora che abbiamo la formula della pareto optimality non ci resta che applicarla ad un esempio concreto. 

\begin{verbatim}
	LET N be SET players
	Una volta stabilito il numero dei giocatori, tutte le tuple 
	hanno lunghezza |N|=:n
	n lo utilizziamo solo per il numero dei giocatori
	altri indici utilizzati saranno: i, j, d
	LET A be SET alternatives
	LET S be SET strategies WHERE s \in S = (a_1, ..., a_n) - 'a' for alternative(choice)
	LET P be SET prospects(outcomes) WHERE p \in P = (p_1, ..., p_n) - 'p' for payoff
	LET u be FUNCTION u : S --> P
	E.G. u(s) |--> (p_1, ..., p_n), p_1, ..., p_n \in R^n
	
	l'operatore su tuple t_{-i} restituisce una tupla di lunghezza di t meno 1.
\end{verbatim}

\begin{definizione}
	Una strategia si dice \emph{pareto-optimally} if there is no other strategy that adhere with the following conditions:
	\begin{enumerate}
		\item 
	\end{enumerate}
\end{definizione}



