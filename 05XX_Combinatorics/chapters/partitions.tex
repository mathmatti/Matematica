\chapter{Partitions of sets}

\section{Partitions, Their Properties, and Their Graphical Representation. \cite{vonNeumann1944}}
Let a set $\Omega$ and a system of sets $\mathcal{A}$ be given. We say that $\mathcal{A}$ is a \emph{partition in $\Omega$} if it fulfills the two following requirements:\\

\medskip
(8:B:a) Every element $A$ of $\mathcal{A}$ is a subset of $\Omega$, and not empty.\\

(8:B:b) $\mathcal{A}$ is a system of pairwise disjunct sets.\\
\medskip

Attenzione perch\`e i requirements sopra potrebbero differire dalla norma. Infatti: 1. Gli insiemi potrebbero anche essere vuoti, 2. La somma di tutti gli elementi dei sottoinsiemi deve dare tutto l'insieme partizionato, qui non \`e richiesto.

This concept too has been the subject of an extensive literature. Cf. \emph{Garrett Birkhoff}: Lattice Theory, New York 1940. This book is of wider interest for the understanding of the modern abstract method. Chapt. VI. deals with Boolean Algebras. Further literature is given there.

\section{Definizioni di partizione}

\begin{definizione}
	A set partition $\pi$ of a set $S$ is a collection $B_1, B_2, ...., B_k$ of nonempty disjoint subsets of $S$ such that $\cup_{i=1}^k B_i = S$. The elements of a set partition are called \emph{blocks}, and the size of a \emph{block} $B$ is given by $|B|$ the number of elements in $B$. We assume that $B_1, B_2, ..., B_k$ are listed in increasing order of their minimal elements, that is, min $B_1 <$ min $B_2 < ... $ min $B_k$. The set of all set partitions of $S$ is denoted by $\mathcal{P}(S)$. \cite{mansour01}
\end{definizione}