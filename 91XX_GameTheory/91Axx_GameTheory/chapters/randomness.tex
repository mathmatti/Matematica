\chapter{Randomness}
\begin{itemize}
 \item The next step in the study of non-cooperative games is the understanding of strategies. As we shall see, the famous Nash equilibrium surely exists only when players $i \in N$ may each randomize over their finite strategy sets $\strategySets$.
\end{itemize}

\section{Discrete random variables: lotteries}
\begin{itemize}
 \item ...
 \item By the way, also recall that a probability distribution over $X$ is defined as any set function $\setFunction$ satisfying $p(A)+p(B)=p(A \cap B)+p(A \cup B)$ for any two events or subsets (of elementary mutually exclusive events) $A, B \in 2^X$, as well as $p(X)=1$, $p(\o)=0$; then, a main theorem on valuations of distributive lattices (such as Boolean lattice $2^X, \cap, \cup)$) $[1, p. 190]$ entails $p(A) = \sum_{i \in A}p(\{i\})$ for all $A \in 2^X$ (this will be detailed when dealing with the \emph{solution} of coalition (cooperative) games $v:2^N \to R$, see below).
     \begin{enumerate}
      \item Che significa $p(A)+p(B)=p(A \cap B)+p(A \cup B)$?
      \item $[1, p. 190]$ cosa dovrei trovare? non ho capito?
     \end{enumerate}

\section{Probabilities, set functions and voting games}
	\begin{itemize}
		\item Recall that the quantitative notion of probability is associated with \emph{events or subsets $A \in 2^X$ of elementary, mutually exclusive (atomic) events.} In 
	\end{itemize}

\end{itemize}