\chapter{Types of cooperative games}
\begin{itemize}
	\item Although in the 70s attention has also been placed on cooperative games with a continuum of players in terms of measure theory (see [5] and related literature), nowadays cooperative games are for the most part dealt with in terms of a finite player set, usually denoted by $N = \{1,...,n\}$. In particular, these games are approached through discrete mathematics as poset/lattice functions. That is, as real-valued functions defined on finite ordered structures. \\
			\begin{enumerate}
					\item Continuum of players vuol dire insieme infinito di giocatori oppure insieme finito/infinito di giocatori che può crescere fino all'infinito? 
					\item Nel caso moderno, cio\`e attraverso l'utilizzo di insiemi finiti e ordinati, nei ragionamenti l'insieme iniziale dei giocatori rimane fisso oppure può crescere?
					\item Measure theory? A measure is a generalization of the concepts of length, area, and volume. 
					\item In [5] what's "value concept"?
					\item Edgeworthian? 
					\item Per il momento questo mi basta!!! Il punto introduce quello che servir\`a sapere in seguito: ordered set/lattices e funzioni definite a partire da questi insiemi all'insieme dei numeri reali. Per ulteriori approfondimenti sui giochi with a continuum of players vedi [5] e letteratura affine.
					\item \quote{Since about 1960, attention has focused more and more on games with large masses of players, in which no individual player can affect the overall outcome. Such games arise naturally in the social sciences, as models for situations in which there are large numbers of very "small" individuals, like consumers in an economy or voters in an election. Mathematically, it is often convenient to represent these games with the aid of a "continuum" of players - like the continuum of points on a line or the continuum of drops in a liquid. Represented thus, such games are called \item{non-atomic}.[5]}
					\item Quindi i nostri giochi sono atomici?
			\end{enumerate}
	
	\item Historically, the first cooperative games were defined in 1953 [27] as set functions $v : 2^N \to R_+,v(0)=0$, with subsets $A \in 2^N$ referred to as coalitions (of players). These games may thus be called coalition games, although in many articles and books they are simply named cooperative games, as if exhausting the whole class of cooperative games.
			\begin{enumerate}
				\item Coalition games sono un tipo di cooperative game?
				\item $2^N$ ? qual \`e il significato di questa notazione?
				\item Prospetto?
				\item Essential games?
			\end{enumerate}
	
	\item Subsequently, in 1963, a further type of cooperative games entered the picture, involving partitions of players or coalition structures, i.e. partitions $P = \{A_1, ..., A_{|P|}\}$ of $N$. In particular, these second-generation cooperative games were named games in partition function form, and they are real-valued functions defined on pairs $(A, P)$ such that $A \in 2^N$ and $P$ is a partition of $N$ such that $A \in P$. These pairs $(A,P)$ are now referred to as "embedded coalitions" (or "embedded subsets" $[13, 14]$). These games pose serious problems in terms of lattice theory, as the corresponding ordered structure (i.e. of embedded subsets) currently needs ad hoc techniques for yielding a lattice (which in any case is not a geometric one, see below). 
			\begin{enumerate}
				\item ...
			\end{enumerate}
		
	\item Finally, in 1990, a third type of cooperative games was introduced and named "global games" $[12]$. These are simply real-valued partition functions, but still lead to embarassing results when it comes to define and quantify the so-called "solution". Roughly speaking, a solution of a cooperative games should determine the a priori worth, for each player, of playing the game. Somehow overcoming the mainstream literature, in the sequel we shall interpret solutions of cooperative games (of any kind) in terms M\''{o}bius inversion and atomic/geometric lattices.
			\begin{enumerate}
				\item "of any kind" si rifesce ai tre tipi di giochi cooperativi o a tutti i giochi, sia cooperativi che non cooperativi?
			\end{enumerate}	
\end{itemize}