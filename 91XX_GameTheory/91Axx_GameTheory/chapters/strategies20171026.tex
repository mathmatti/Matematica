\chapter{Strategies v. 26/10/2017}
Il concetto di strategia varia a seconda del tipo di gioco. Cosa vuol dire che il concetto varia? Vuol dire che a seconda del gioco \`e rappresentato da un certo tipo di oggetto matematico. 
Quasi sempre comunque la strategia \`e un elemento di un insieme pi\`u o meno complesso. Quindi secondo questo ragionamento l'alternative set, strategy set, action profiles, etc. sono tutte strutture matematiche che possono essere annoverate tra quelle che rappresentano il concetto di strategia del mondo reale ed i cui elementi sono, appunto, strategie. \\

Cominciamo con gli appunti del Prof. Giovanni Rossi (\cite{rossi26102017}).

\begin{itemize}
	\item In \emph{simultaneous-move games} alla players move only once, simultaneously, hence choosing a strategy is the same as choosing a move. This is no longer true in \emph{multistage games}, where choosing a strategy means choosing a \emph{sequence of (conditional) moves}. Although the non-cooperative games to be dealt with shall be in simultaneous-move form, still multistage games are briefly described below in order to formally define strategies in a most general setting, namely where players have either perfect or else incomplete information, this latter being commonly modeled by means of partitions.
\end{itemize}

\section{Information in multistage games}

\begin{itemize}
	\item As the name clearly suggests, multistage games are played in discrete time $t = 0, 1, ..., T$, as $t = 0$ is the starting point or \emph{root of the game tree} (defined hereafter), where some (at last one, and possibly all) players move; next, depending on previous moves, at each $t \ge 1$ a \emph{node} is reached, corresponding either to a moment where at least one player has to move, or else to an end of the game or \emph{leaf}. The concern is only with games where $T < \infty$ (for any leaf).
	
	\item Multistage games are thus commonly represented by a \emph{rooted and directed (game) tree} $\gameTree$, $\vertexSet = \vertexList$
\end{itemize}




\section{Dominated and dominant strategies}
\section{Deletion of dominated strategies}
\section{Equilibrium}