\chapter{Potential games}
Un gioco a potenziale, o gioco con potenziale, \`e un gioco in cui l'incentivo per i giocatori per passare da una strategia ad un'altra pu\` essere espresso con una singola funzione globale, detta funzione potenziale, richiamando l'omonimo concetto fisico.

Il concetto fu introdotto da Dov Monderer e Lloyd Shapley nel 1996. Vedi \cite{2}.

La funzione potenziale si rivela uno strumento utile per analizzare gli equilibri di Nash in certi giochi, dato che gli incentivi di tutti i giocatori sono mappati in una singola funzione, e l'insieme degli equilibri di Nash si trova fra gli ottimi locali della funzione potenziale.

I massimi della funzione potenziale sono equilibri di Nash, mentre l'inverso non \`e sempre vero. L'uso dei massimi della funzione potenziale permette di raffinare l'insieme degli equilibri di Nash.


il \cite{2} \`e un tantino complesso da leggere, pertanto inizierei da qualcosa di more simple.

...e congestion games...

\section{Congestion games}

%%\section{CHAPTER TODO}
%%Section 19 in: Vazirani, Vijay V.; Nisan, Noam; Roughgarden, Tim; Tardos, Éva (2007). Algorithmic Game Theory (PDF). Cambridge, UK: Cambridge University Press. ISBN 0-521-87282-0.