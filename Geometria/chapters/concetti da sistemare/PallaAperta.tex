\documentclass[a4paper,10pt]{article}
\usepackage[utf8]{inputenc}
\usepackage{amsmath}
\usepackage{hyperref}

%opening
\title{Palla aperta (o disco aperto)}
\author{\href{http://www.baudo.hol.es}{giuseppe baudo}}

\begin{document}

\maketitle

\section{Definizione}
Si chiama \textit{palla aperta} dello spazio metrico $(S, d)$ di \textit{centro} $x$ e \textit{raggio} $r$ l'insieme così definito:
\[
B(x, r) = \{z \in S : d(z, x) < r\}
\]

\section{Note}
Questo genere di insieme è possibile solo se parliamo di spazi metrici ovvero in qualche modo riusciamo a definire una funzione che ci restituisca il valore di $r$.
Altra cosa da considerare è il simbolo $<$ in qualche modo questo implica che $<$ stesso sia definito nell'ambito dell'insieme $S$.

\end{document}
