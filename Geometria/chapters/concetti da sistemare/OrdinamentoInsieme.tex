\documentclass[a4paper,10pt]{article}
\usepackage[utf8]{inputenc}
\usepackage{amsmath}
\usepackage{hyperref}

%opening
\title{Ordinamento di un insieme}
\author{\href{http://www.baudo.hol.es}{giuseppe baudo}}

\begin{document}

\maketitle

\section{Definizione}
Un \textit{ordinamento} in un insieme $X$, o \textit{relazione d'ordine}, o \textit{lemma di Zorn}, è una relazione che soddisfa queste proprietà:
\begin{itemize}
 \item Riflessiva: $x \leq x \text{per ogni} x \in X$
 \item Antisimmetrica: $\text{se } x \leq y \text{ e } y \leq x, \text{ allora } x=y$
 \item Transitiva: $\text{se } x \leq y \text{ e } y \leq z, \text{ allora } x \leq z$
\end{itemize}




\end{document}
