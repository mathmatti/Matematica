\section{Equazione Cartesiana del Piano}
\begin{definizione}
Ogni equazione lineare in $x,y e z$ del tipo $ax+by+cz+d=0$ rappresenta, a meno di un fattore moltiplicativo non nullo,
l'equazione cartesiana di un piano nello spazio $S_3$ (chi essere $S_3$???).
\end{definizione}

\begin{osservazione}
In realtà qui la questione riguarda un teorema il cui risultato è utilizzatissimo nelle applicazioni pratiche.

Si dimostrerà che ogni equazione di primo grado in $x$, $y$ e $z$ del tipo:
\[
 ax+by+cz+d=0
\]

con $a,b,c,d \in R$ e $a,b,c$ non contemporaneamente tutti uguali a zero, $(a,b,c) \ne (0,0,0)$ rappresenta un piano. 
Viceversa, ogni piano dello spazio è rappresentabile tramite un'equazione lineare in $x,y,z$ del tipo suddetto.
\end{osservazione}



