\section{Geometria Euclidea}
La geometria Euclidea bla bla bla
%\section{Geometria Euclidea: definizione}
\begin{definizione}
Si chiama \emph{\bf{geometria euclidea}} la classica geometria che si studia alle scuole elementari, medie e superiori. E' quella che si basa sugli Elementi di Euclide dove
i concetti fondamentali (ovvero gli assiomi) sono quelli di punto, retta, etc.
\end{definizione}

\begin{definizione}
Si chiama \emph{\bf{segmento orientato (non banale)}} un segmento su cui sia stato scelto un verso, ovvero su cui si sia fissato un ordine tra i due estremi. 
\end{definizione}

\begin{definizione}
Si chiama \emph{\bf{segmento orientato banale}} \`e quello che inizia in un punto $A$ e finisce sempre in $A$.
\end{definizione}

\begin{definizione}
Si chiama \emph{\bf{insieme dei segmenti orientati}} l'insieme dei segmenti orientati.
\end{definizione}

\begin{osservazione}
Ovviamente l'insieme dei segmenti orientati \`contiene anche il segmento orientato banale per definizione.
\end{osservazione}

\begin{definizione}
L'\emph{\bf{equipollenza}} \`e una relazione di equivalenza tra due segmenti orientati dell'insieme dei segmenti orientati.
\end{definizione}

\begin{definizione}
Si chiama \emph{\bf{vettore geometrico dello (o nello) spazio}}
\end{definizione}

\section{Vettore}
\subsection{Vettore colonna, vettore riga}
Un vettore in uno spazio $n$-dimensionale è un insieme ordinato formato da $n$ valori. 

\subsection{Vettore in geometria mono, bi e tri-dimensionale}
Un vettore è un oggetto che ha una direzione e una lunghezza. In questo caso si dimostrerà che un vettore può essere
rappresentato come da definizione precedente.

\subsection{Componenti di un vettore}

\subsection{Rappresentazione canonica}

\subsection{Lunghezza di un vettore in $R^n$}
The length of a vector $v$ in $R^n$ is the square root of the sum of the squares of tis components.
\[
 |v|=\sqrt{v^2_1+...+v^2_n}
\]

This is a natural generalization of the Pythagorean Theorem.

\subsection{dot product or scalar product}
The dot product (or inner product or scalar product) of two $n$-component real vectors is the linear combination of their components.
\[
 u \dotfill v = u_1v_1+...+u_nv_n
\]
squares of its components.


\subsection{NOTAZIONE}

\subsection{NOTE}
Le due definizioni sono equivalenti nel senso che si possono rappresentare i vettori della definizione 2 come vettori
della definizione 1.

Attenzione alla definizione di vettore libero.

Attenzione all'uguaglianza tra due vettori. Due vettori sono uguali quando hanno la stessa rappresentazione canonica.



