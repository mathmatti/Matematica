\documentclass[a4paper,10pt]{article}
\usepackage[utf8]{inputenc}

%opening
%\title{}
%\author{}


\begin{document}

%\maketitle
%\tableofcontents
%\newpage

\section{Neighborhood of a point}
\subsection{Definition}


Let $ X $ be a metric space. All points and sets mentioned below
are understood to be elements and subsets of $ X $.

A \textit{neighborhood} of $ p $ is a set $ N_{r}(p) $ consisting of all $ q $ such that
$ d(p,q) < r $, for some $ r > 0 $. The number $ r $ is called the \textit{radius} of $ N_{r}(p) $.

\subsection{Remarks}

Cosa significa for some $ r > 0 $? Penso non voglia dire nulla, in caso l'insieme potrebbe contenere solo il punto $ p $\\
Quando ho bisogno di usare questa definizione la scrivo nella seguente forma: \\
$ N_{r}(p)=\{q \in X : d(p,q)<r, r>0 \} $ \\
L'insieme $ N_{r}(p) $ deve necessariamente contenere $ p $? \\
non è richiesto esplicitamente che $ q $ deve essere diverso da $ p $ per cui se $ q=p \rightarrow d(p,p)=0 < r $

\end{document}
