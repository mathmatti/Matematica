\documentclass[a4paper,10pt]{article}
\usepackage[utf8]{inputenc}
\usepackage{amsmath}
\usepackage{hyperref}

%opening
\title{Metric Space}
\author{\href{http://www.baudo.hol.es}{giuseppe baudo}}

\begin{document}

\maketitle

\section{Definizione (1)}
Uno spazio metrico è la coppia $(S, d)$, dove $S$ è un insieme e $d$ una \href{Distanza.html}{distanza}.

\section{Definizione (2)}
Uno spazio metrico è un insieme $S$ in cui (o con cui) è possibile definire una funzione \href{Distanza.html}{distanza}.

\section{Symbols}
\begin{itemize}
 \item $(S, d)$ - Spazio metrico $S$ con la distanza $d$
 \item $(C, d)$ - Spazio metrico $C$ dei numeri complessi con la distanza $d$
 \item $S$ - generico spazio metrico $S$ 
 \item $X$ - generico spazio metrico $X$  
\end{itemize}

\section{Esempio}
L'insieme R dei numeri reali con la funzione di distanza geometrica è uno spazio metrico.

\section{Note}
Gli elementi che compongono il concetto di spazio metrico sono due: un insieme ed una funzione che soddisfi determinate proprietà. Una caratteristica che qui segnalo ma che dovrebbe essere oggetto di approfondimento
è il fatto che utilizzando lo stesso insieme è probabile che si riesca a costruire più di uno spazio metrico semplicemente segliendo due distanze distinte.

\begin{itemize}
 \item Spazio metrico 1: $(S, d_1)$ con $d_1=corpoDellaFunzione$
 \item Spazio metrico 2: $(S, d_2)$ con $d_2=corpoDellaFunzione$
\end{itemize}


\end{document}
