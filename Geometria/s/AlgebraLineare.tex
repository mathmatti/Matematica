\section{Algebra Lineare}



\section{Introduzione tratta dal libro "Introduzione all'algebra lineare" Prof. Fioresi-Morigi}
Le applicazioni lineari sono funzioni tra spazi vettoriali che ne rispettano la struttura, cioè sono compatibili con le operazioni di somma tra vettori e moltiplicazione di un
vettore per uno scalare. Come vedremo le applicazioni lineari si rappresentano in modo molto efficace attraverso le matrici. Lo scopo di questo capitolo è quello di introdurre
il concetto di applicazione lineare e capire come sia possibile associare univocamente una matrice ad ogni applicazione lineare tra $R^n$ e $R^m$, una volta fissata in entrambi
gli spazi la base canonica. Studieremo poi il nucleo e l'immagine di un'applicazione lineare fino ad arrivare al Teorema della dimensione, che rappresenta uno dei risultati più
importanti della teoria sugli spazi vettoriali di dimensione finita.




\subsection*{Syllabus Algebra Lineare}
\subsubsection{Spazi Vettoriali}
\begin{itemize}
 \item Vettore
 \item \href{CoordinateVettore.pdf}{Coordinate di un vettore}
 \item \href{SpazioVettoriale.pdf}{Spazio Vettoriale}
 \item Sottospazio vettoriale
 \item Combinazione lineare
 \item Spazio generato dai vettori $v_1, ..., v_k$
 \item Sistema di generatori
 \item Versore
 \item Interdipendenza lineare (Indipendenza lineare)
 \item \href{Base.pdf}{Base di uno spazio vettoriale}
 \item Base canonica
 \item Matrice di cambiamento di base
 \item Criterio di indipendenza
 \item Estrazione di una base
 \item Completamento a una base
 \item \href{Dimensione.pdf}{Dimensione di uno spazio vettoriale}
 \item Componenti
 \item Somma diretta e somma 
 \item Spazi quozienti
 \item Duale
\end{itemize}

\subsubsection{Matrici}
\begin{itemize}
   \item \href{Matrice.pdf}{Matrice}
   \item Matrici identit\'{a} (identica)
   \item Matrice nulla
   \item Matrice opposta 
   \item \href{InsiemeDelleMatrici.pdf}{Insieme delle matrici $m \times n$: $M_{m,n}(K)$}
   \item Matrice trasposta
   \item Properties of Transpose Matrices: \url{http://www.math.nyu.edu/~neylon/linalgfall04/project1/dj/proptranspose.htm}
   \item Help with proving that the transpose of the product of any number of matrices is equal to the product of their transposes in reverse.
   \item La trasposta della trasposta \'{e} la matrice stessa
   \item La trasposta della somma di due matrici è uguale alla somma delle due matrici trasposte
   \item L'ordine delle matrici si inverte per la moltiplicazione
   \item Se $c$ \'{e} uno scalare, la trasposta di uno scalare \'{e} lo scalare invariato
   \item Nel caso di matrici quadrate, il determinante della trasposta \'{e} uguale al determinante della matrice iniziale
   \item La trasposta di una matrice invertibile \'{e} ancora invertibile e la sua inversa \'{e} la trasposta dell'inversa della matrice iniziale
   \item Se $A$ \'{e} una matrice quadrata, allora i suoi autovalori sono uguali agli autovalori della sua trasposta
   \item \href{SommaMatrici.pdf}{Operazione di somma tra matrici}
   \item \href{ProdottoMatrici.pdf}{Operazione di prodotto tra matrici}
   \item Properties of matrix multiplication: \url{https://www.khanacademy.org/math/precalculus/precalc-matrices/properties-of-matrix-multiplication/a/properties-of-matrix-multiplication}
   \item Operazione di prodotto tra uno scalare ed una matrice
   \item \href{MatriceQuadrata.pdf}{Matrice Quadrata}
   \item \href{OrdineMatrice.pdf}{Ordine di una matrice quadrata}
   \item \href{DeterminanteMatrice.pdf}{Determinante di una matrice quadrata}
   \item \href{PolinomioMatrice.pdf}{Polinomio caratteristico di una matrice}
   \item \href{AutovaloriMatrice.pdf}{Autovalori di una matrice quadrata}
   \item \href{Autovalore.pdf}{Autovalore}
   \item Autovettore
   \item Molteplicità algebrica e geometrica
   \item Matrice diagonale
   \item \href{MatriceDiagonalizzabile.pdf}{Matrice diagonalizzabile}
\end{itemize}

\subsubsection{Sistemi lineari}
  \begin{itemize}
   \item Sistema di che cosa?
   \item Sistema lineare e matrici.
   \item Sistema lineare
   \item Sistema omogeneo
   \item Risoluzione
  \end{itemize}
  
\subsubsection{Applicazioni lineari}
  \begin{itemize}
   \item \href{IntroAppLineari.pdf}{Introduzione alle applicazioni lineari}
   \item Applicazione lineare
   \item Nucleo di un'applicazione lineare
   \item Immagine di un'applicazione lineare
   \item Base di un'applicazione linearea
   \item Cambio di base per un'applicazione lineare
   \item \href{ApplicazioneIniettiva.pdf}{Applicazione lineare iniettiva}
   \item \href{DimensioneImmagine.pdf}{Dimensione dell'immagine di un'applicazione lineare}
   \item Teorema della dimensione
   \item \href{MatriceApplicazione.pdf}{Matrice associata ad un'applicazione lineare tra spazi vettoriali}
   \item Isomorfismo di applicazioni lineari
   \item Calcolo del nucleo e dell'immagine
   \item Diagonalizzazione
   \item Applicazione lineare diagonalizzabile
  \end{itemize}


\section{DEFINIZIONE}
Un'applicazione lineare è iniettiva se il nucleo dell'applicazione è il vettore nullo.



\section{DEFINIZIONE}
La dimensione dell'imagine di un'applicazione lineare è uguale al rango 
della matrice associata all'applicazione lineare.



\section{DEFINIZIONE}
La matrice associata ad un'applicazione lineare tra spazi vettoriali si ottiene mettendo nelle colonne della matrice
le immagini dei vettori della base canonica del dominio (o di altra base data).

\section{NOTAZIONE}

\section{ESEMPIO}

\section{APPROFONDIMENTI}
\begin{itemize}
 \item \url{http://www.dm.unibo.it/~ida/NoteGeometria1-25-5-16.pdf}
 \item \url{http://joshua.smcvt.edu/linearalgebra/book.pdf}
 \item \url{http://www.youmath.it/lezioni/algebra-lineare/applicazioni-lineari/771-calcolare-dimensione-e-base-di-nucleo-e-immagine.html}
\end{itemize}


\section{Matrice}
$A = (a_{ij})_{i=1...m,j=1...n}$

la seguente notazione ritengo sia errata:
$A = a_{ij}$

Perch\'{e} stiamo cerchando di assegnare un elemento di una matrice ad una matrice.

Pertanto sarebbe corretto scrivere: $A = (a_{ij})$ oppure in maniera completa $A = (a_{ij})_{i=1...m,j=1...n}$


\section{DEFINIZIONE}
Matrice che ha lo stesso numero di righe e colonne.



\section{DEFINIZIONE}
L'ordine di una matrice quadrata è il numero di righe o equivalentemente il numero di colonne.

\section{ESEMPIO}
Una matrice quadrata con n righe ed e colonne ha ordine n.



\section{DEFINIZIONE}
Sia A una \href{./MatriceQuadrata.html}{matrice quadrata}, t una incognita (la nostra x) e $I_n$ la matrice identità di \href{./OrdineMatrice.html}{ordine} n. Il polinomio caratteristico (nella incognita $t$) di una matrice è uguale al \href{./DeterminanteMatrice.html}{determinante} della matrice:
\[
 A-tI_n
\]

scriviamo:
\[
 p_A(t) := det(A-tI_n)
\]



\section{Introduzione}
Sotto talune condizione riguardo ai tipi di matrice, cio\'{e} sul numero delle righe e delle colonne che le matrici che si intendono moltiplicare devono avere, definiamo un'operazione
che chiamiamo \textit{operazione di prodotto tra matrici} o \textit{prodotto tra matrici} o \textit{prodotto riga per colonna}.
Bene, diciamolo subito, tra tutte le matrici dell'insieme $M_{m,n}$ \'{e} possibile moltiplicare tra di loro quelle in cui il numero di colonne della prima matrice corrisponde al
numero di righe della seconda matrice. Per questo motivo viene chiamato prodotto righe per colonne.

Altra cosa da tenere presente \'{e} il fatto che la forma della matrice risultato pu\'{o} essere diversa da quella delle matrici che abbiamo moltiplicato. Ed infatti, la matrice risultato
sar\'{a} composta da $m$ righe ed $n$ colonne, dove, $m$ \`{e} il numero di righe della prima matrice e $n$ \`{e} il numero di colonne della seconda matrice.

\begin{definizione}
 Se $A$ \`{e} una matrice $m \times s$ e $B$ \`{e} una matrice $s \times n$, definiamo il prodotto $c_{ij}$ della riga $i$ di $A$ e della colonna $j$ di $B$ nel modo seguente:
 \[
  c_{ij} = \begin{pmatrix}
	      a_{i1} & ... & a_{is}
           \end{pmatrix}           
           \begin{pmatrix}
	      b_{1j} \\
	      ... \\
	      b_{sj}
           \end{pmatrix}
         = a_{i1}b_{1j} + ... + a_{is}b_{sj}
         = \sum_{h=1}^s a_{ih}b_{hj}
 \]

\end{definizione}

In altre parole, per ricavare l'elemento di posto $(i,j)$ bisogna eseguire il prodotto riga per colonna tra la riga $i$ e la colonna $j$.

Analogamente all'operazione di somma tra matrici, quello che \`{e} importante capire \`{e} quale formula \`{e} utilizzata per ricavare l'elemento di posto $(i,j)$. Solo che,
nel caso della somma tra matrici, l'operazione \`{e} banale ed immediata, qui invece l'operatore di sommatoria potrebbe dare qualche problema agli inizi, ma poi ci si fa l'abitudine.



\section{DEFINIZIONE}
L'operazione di somma tra matrici si effettua applicando l'operazione di somma del campo $K$ di cui sono fatti gli elementi delle matrici.
Le matrici devono avere lo stesso numero di righe e di colonne.

\section{SIMBOLI}
$A = B + C$

$a_{ij} = b_{ij} + c_{ij}$





\section{DEFINIZIONE}
Gli autovalori di una matrice quadrata sono le radici del polinomio caratteristico.




\section{DEFINIZIONE}
L'insieme delle matrici $mxn$: $M_{m,n}(K)$ ha una struttura di spazio vettoriale con l'operazione di somma tra matrici.


\section{Definizione (1)}
Una matrice quadrata $A$ si dice \textit{diagonalizzabile} se esiste una matrice invertibile $P$ tale che $P^{-1}AP$ sia diagonale.

\section{Definizione (2)}
Una matrice è diagonalizzabile se:
\begin{enumerate}
 \item Il numero degli autovalori, contati con la loro molteplicità, sia pari all'ordine della matrice.
 \item La molteplicità geometrica di ciascun autovettore coincida con la relativa molteplicità algebrica.
\end{enumerate}

\section{APPROFONDIMENTI}
\begin{itemize}
 \item \url{http://www.youmath.it/lezioni/algebra-lineare/matrici-e-vettori/1581-matrice-diagonalizzabile.html}
\end{itemize}



\section{DEFINIZIONE}
Si chiama Spazio Vettoriale una qualunque struttura matematica che possiede determinate propriet\'{a}.

\section{Descrizione}
Affinch\'{e} si possa parlare di spazio vettoriale occorrono almeno due insiemi generalmente indicati con $V$ e $K$. Tra gli elementi di $V$ \'{e} definita
una funzione che associa ad ogni coppia di elementi di $V$ un elemento di $V$. Tale funzione viene chiamata operazione di somma.
\section{Propriet\'{a}}


\section{ESEMPIO}
L'insieme dei numeri reali con l'operazione di somma con l'operazione di somma e prodotto.
L'insieme dei vettori geometrici dello spazio con l'operazione di somma tra vettori e prodotti di vettori per uno scalare.




\section{DEFINIZIONE}
A basis for a vector space is a sequence of vectors that is linearly independent and that spans the space.

\section{NOTE}
A vector space can have many different bases.

Tutte le basi hanno lo stesso numero di vettori.



\section{Definizione}
buona bibliografia qui: \url{https://it.wikipedia.org/wiki/Coordinate_di_un_vettore}




\section{DEFINIZIONE}
The dimension of a vector space is the number of vectors in any of its bases.


