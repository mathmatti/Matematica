\documentclass[a4paper,10pt]{article}
\usepackage[utf8]{inputenc}
\usepackage{amsmath}
\usepackage{hyperref}

%opening
\title{Insieme aperto}
\author{\href{http://www.baudo.hol.es}{giuseppe baudo}}

\begin{document}

\maketitle

\section*{Definizione}
Sia $(S, \bar{U})$ uno \href{SpazioTopologico.html}{spazio topologico}. Si dice che un insieme $A \subseteq S$ è un \textit{insieme aperto} se $A = \{\}$ oppure $A \ne \{\}$ ed ogni
punto di $A$ possiede almeno un intorno contenuto in $A$:
\[
 \forall x \in A \exists U \in \bar{U(x)} : U \subseteq A
\]



\end{document}
