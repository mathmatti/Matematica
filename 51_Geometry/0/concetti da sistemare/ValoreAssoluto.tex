\documentclass[a4paper,10pt]{article}
\usepackage[utf8]{inputenc}
\usepackage{amsmath}
\usepackage{hyperref}

%opening
\title{Valore assoluto}
\author{\href{http://www.baudo.hol.es}{giuseppe baudo}}

\begin{document}

\maketitle

\section{DEFINIZIONE}
Il valore assoluto o modulo di un numero reale $x$ è una funzione che associa a $x$ un numero reale non negativo. Se $x$
è un numero reale, il suo valore assoluto è $x$ stesso se $x$ è non negativo, e $-x$ se $x$ è negativo.

\section{NOTAZIONE}
Il valore assoluto di un numero $x$ si indica con $|x|$

\section{ESEMPIO}
Il valore assoluto sia di $3$ che di $-3$ è $3$. 

\section{PROPRIETA'}
Sono molto imporanti nella pratica ti consiglio di dare un occhio alla sezione approfondimenti per saperne di più.

\section{APPROFONDIMENTI}
\begin{itemize}
 \item \url{https://it.wikipedia.org/wiki/Valore_assoluto}
\end{itemize}

\end{document}
