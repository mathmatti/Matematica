\documentclass[a4paper,10pt]{article}
\usepackage[utf8]{inputenc}
\usepackage{amsmath}
\usepackage{hyperref}

%opening
\title{Topologia}
\author{\href{http://www.baudo.hol.es}{giuseppe baudo}}

\begin{document}

\maketitle

\section*{Syllabus}
\subsection*{Metric Spaces}
  \begin{itemize}
    \item \href{Distanza.pdf}{Distanza}
    \item \href{MetricSpace.pdf}{Metric Space}
    \item \href{PointMetricSpace.pdf}{Point of a Metric Space}
    \item \href{SubsetMetricSpace.pdf}{Subset of a Metric Space}
    \item \href{DiametroSpazioMetrico.pdf}{Diametro di uno spazio metrico} 
    \item Spazio metrico limitato (è quello con diametro finito).
    \item Spazio metrico illimitato (è quello con diametro infinito). 
    \item \href{PallaAperta.pdf}{Palla aperta (o disco aperto) di uno spazio metrico (se l'intervallo non contiene gli estremi)}
    \item Palla chiusa (o disco chiuso) di uno spazio metrico (se l'intervallo contiene gli estremi)
    \item Insiemi aperti, chiusi (o spazi metrici aperti e chiusi? vedi sotto topologia). 
    \item Unione e intersezione di insiemi aperti o chiusi. 
    \item Definizione di interno, chiusura e frontiera di un insieme e loro proprietà. 
    \item Spazi metrici e insiemi connessi. 
    \item Insiemi connessi in R. 
    \item Poligonale.
    \item Insiemi aperti connessi in C. 
    \item Successioni convergenti, punti limite. 
    \item La chiusura di un insieme coincide con i suoi punti limite. 
    \item Insiemi densi. 
    \item Successioni di Cauchy.
    \item Le successioni convergenti sono di Cauchy. 
    \item Una successione di Cauchy che ammette una sottosuccessione convergente è convergente. 
    \item Spazi metrici e insiemi completi.
    \item Completezza di C (assumendo R completo). 
    \item Un sottoinsieme di uno spazio metrico completo è completo se e solo se è chiuso. 
    \item Spazi metrici e insiemi (sequenzialmente) compatti. 
    \item Uno spazio metrico compatto è completo. 
    \item Spazi metrici totalmente limitati. 
    \item Uno spazio metrico totalmente limitato è limitato. 
    \item Uno spazio metrico è compatto se e solo se è completo e totalmente limitato. 
    \item Un sottoinsieme di R n è compatto se e solo se è chiuso e limitato. 
    \item Intorno
    \item \href{LimitPoint.pdf}{Limit point (Punto di accumulazione)}
    \item \href{InsiemeCompatto.pdf}{Insieme compatto}
  \end{itemize}  

  \subsection*{Euclidean spaces}
    \begin{itemize}
     \item 
    \end{itemize}

  
  \subsection*{Topologia generale}
    \begin{itemize}
    \item \href{Topologia.pdf}{Introduzione alla Topologia Generale}
    \item \href{SpazioTopologico.pdf}{Spazio topologico}
    \item Sottospazio di uno spazio topologico.
    \item \href{IntornoPunto.pdf}{Intorno di un punto} 
    \item \href{InsiemeAperto.pdf}{Insieme aperto}
    \item Insieme chiuso.
    \item Punto interno di un insieme.
    \item Insieme dei punti interni di un insieme.
    \item Confronto di topologie. 
    \item Chiusura, parte interna, frontiera di un sottoinsieme. 
    \item Sottoinsiemi densi. 
    \item Assiomi di numerabilita'. 
    \item Funzioni continue e omeomorfismi. 
    \item Topologia indotta su un sottoinsieme, prodotto di spazi topologici e quoziente di uno spazio topologico.  
    \item Assioma di Hausdorff.  
    \item Connessione, componenti connesse di uno spazio topologico. 
    \item Locale connessione. 
    \item Compattezza, compattezza in spazi metrici. 
    \item Spazi localmente compatti. 
    \end{itemize}
\end{document}
