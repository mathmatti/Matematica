
\section{Definizione}
Un'\textit{operazione} è una funzione/mappa da un insieme in se stesso. 

\section{Symbol}
Il concetto di operazione, intesa qui in senso astratto ovvero quale rappresentante di tutte le operazioni come quelle degli esempi, a differenza degli esempi, però, non ha un simbolo proprio ma appunto
ogni operazione specifica ha un proprio simbolo, per esempio la somma tra numeri si indica con $+$ mentre il prodotto di numeri si indica con $*$. E' il caso che un'operazione
specifica possa avere più di un simbolo che la rappresenta, vedremo caso per caso.

Quindi, ricapitolando, il $+$ o il $-$ o il $\times$ o il $:$ sono operazioni. Operazione è il concetto che raccoglie a fattor comune tutte le operazioni. La caratteristica comune
è il fatto di essere una funzione/mappa da un insieme in se stesso.

\section{Esempio}
Somma tra numeri naturali, prodotto tra numeri naturali, somma tra numeri reali, somma tra numeri immaginari, somma tra matrici, prodotto di funzioni (operazioni di composizione), etc.

\end{document}
