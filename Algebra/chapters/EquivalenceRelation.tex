\documentclass[a4paper,10pt]{article}
\usepackage[utf8]{inputenc}
\usepackage{amsmath}
\usepackage{hyperref}

%opening
\title{equivalence relation}
\author{\href{http://www.baudo.hol.es}{giuseppe baudo}}

\begin{document}

\maketitle

\section{Definizione}
A relation $\tilde{-}$ on a set $S$ is called an \textit{equivalence relation} if, for all $a, b, c \in S$, it satisfies:
\begin{itemize}
 \item $a \tilde{-} a$ (reflexivity)
 \item $a \tilde{-} b$ implies that $b \tilde{-} a$ (symmetry)
 \item $a \tilde{-} b$, $b \tilde c$ implies that $a \tilde{-} c$ (transitivity)
\end{itemize}

\section{Symbol}
$\tilde{-}$



\section{Esempio}
L'uguaglianza, $=$, è una relazione di equivalenza.

\end{document}
