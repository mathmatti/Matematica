\chapter{Equivalence Relation}

\begin{definizione}
A relation $\tilde{-}$ on a set $S$ is called an \textit{equivalence relation} if, for all $a, b, c \in S$, it satisfies:
\begin{itemize}
 \item $a \tilde{-} a$ (reflexivity)
 \item $a \tilde{-} b$ implies that $b \tilde{-} a$ (symmetry)
 \item $a \tilde{-} b$, $b \tilde c$ implies that $a \tilde{-} c$ (transitivity)
\end{itemize}
\end{definizione}

Symbol: $\tilde{-}$

Esempio: \\
L'uguaglianza, $=$, è una relazione di equivalenza.


