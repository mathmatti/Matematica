\documentclass[a4paper,10pt]{article}
\usepackage[utf8]{inputenc}
\usepackage{amsmath}
\usepackage{hyperref}

%opening
\title{Gruppo generale lineare}
\author{baudo81[at]gmail.com}

\begin{document}

\maketitle

\begin{abstract}
Vediamo di cosa si tratta
\end{abstract}

\section{Definizione}
from wikipedia:
Il gruppo lineare generale è il gruppo di tutte le matrici invertibili $n \times n$ 
a valori in un campo $K$, dove $n$ è un numero intero positivo. 
Il gruppo lineare generale viene indicato con $GL(n, K)$ oppure con $GL_{n}(K)$, e si dice anche gruppo di matrici.

L'insieme $GL(n, K)$ forma un gruppo con l'operazione di \href{./prodottoMatrici.html}{moltiplicazione fra matrici}.

\end{document}
