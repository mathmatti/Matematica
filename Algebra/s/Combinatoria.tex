\begin{definizione}
Sia $X$ un insieme non vuoto. Si dice permutazione su $X$ ogni applicazione bigettiva di $X$ in se stesso. \cite{permutazione1}
\end{definizione}

\begin{osservazione}
NOTAZIONE - FORMA MATRICIALE

In generale, per indicare una permutazione si usano le lettere greche minuscole, es. $\sigma$, e la cosiddetta notazione matriciale,
nella quale sono riportarte (nella seconda riga) le immagini secondo $\sigma$ degli elementi di $X$ (scritti nella prima riga):
\[
\left(
 \begin{array}{cccc}
  1 & 2 & ... & n \\
  \sigma(1) & \sigma(2) & ... & \sigma(n)
 \end{array}
\right)
\] 

\end{osservazione}

\begin{osservazione}
 NOTAZIONE (2) - CICLI
 
see \href{./PermutazioneCicli.html}{Ciclo di una permutazione}
\end{osservazione}


\section{PERMUTAZIONE IDENTICA - ELEMENTO NEUTRO RISPETTO ALLA COMPOSIZIONE DI PERMUTAZIONI}
In questa notazione, l'applicazione identica corrisponde ad una matrice con due righe uguali:
\[
\left(
 \begin{array}{cccc}
  1 & 2 & ... & n \\
  1 & 2 & ... & n \\
 \end{array}
\right)
\]

Indicheremo tale applicazione (detta permutazione identica), più semplicemente, con il simbolo $id$. \cite{permutazione1}

\section{INVERSA DI UNA PERMUTAZIONE}
per ottenere l’inversa di una permutazione basta scambiare la prima e la seconda riga e riordinare la prima. \cite{permutazione2}

\section{INSIEME DELLE PERMUTAZIONI}
Denoteremo con $S(X)$ l'insieme delle permutazioni su $X$. \cite{permutazione1}

Il numero di elementi di $S(X)$ è uguale a $n!$, dove $n$ è il numero di elementi dell'insieme $X$.

\section{NOTE}
Una permutazione è una funzione

\section{APPROFONDIMENTI}
\begin{itemize}
 \item \href{./pdf/PERMUTAZIONE/lezione4.pdf}{DISPENSA: Gruppi di permutazioni} \cite{permutazione1}
 \item \href{./pdf/PERMUTAZIONE/permutazioni.pdf}{DISPENSA: Permutazioni} \cite{permutazione2}
 \item \href{http://www.dm.uniba.it/~barile/Rete4/algebra1_pdf/lezione18.pdf}{DISPENSA: Orbite e cicli di una permutazione.} \cite{permutazione4}
 \item \href{http://progettomatematica.dm.unibo.it/Permutazioni/homepg.htm}{PROGETTO MATEMATICA: Permutazioni} \cite{permutazione3}
\end{itemize}

\section{DEFINIZIONE (1)}
Per ciclo di una permutazione si intende il nome della notazione utilizzata per rappresentare una permutazione.

\section{DEFINIZIONE (2)}
Sia $n$ un intero positivo. Si dice ciclo (o permutazione ciclica) ogni $\sigma \in S_n$ per cui esistono un intero
positivo $l$ e $a_1, ..., a_l \in \{1, ..., n\}$ a due a due distinti tali che
\begin{itemize}
 \item $\sigma(a_1)=a_2, \sigma(a_2)=a_3, ..., \sigma(a_l)=a_1$;
 \item $\sigma(k)=k$ per ogni $k \in \{1, ..., n\} \setminus \{a_1, ..., a_l\}$.
\end{itemize}
Il numero $l$ si dice lunghezza di $\sigma$. Una permutazione ciclica di lunghezza $l$ si dice anche l-ciclo.

\section{DEFINIZIONE (3)}
Sia $r$ un intero positivo, $2 \leq r \leq n $ e siano dati $r$ elementi distinti $i_1, i_2, ..., i_r \in X=\{1,2,...,n\}$.
Col simbolo $\gamma = (i_1 i_2 ... i_r)$ si denoti la permutazione $\gamma \in S_n$ tale che:
\begin{enumerate}
 \item $\gamma (i_k) = i_k$ se $i_k \notin \{i_1, i_2, ..., i_r\}$
 \item $\gamma(i_k)=i_{k+1}$ se $1 \leq k \leq r-1$
 \item $\gamma (i_r)=i_1$
\end{enumerate}
Tale permutazione è detta ciclo di lunghezza $r$.
Se il ciclo ha lunghezza 2 viene detto trasposizione o scambio.

\subsection{NOTE}
Il solo ciclo di lunghezza 1 è la permutazione identica.

Il ciclo di lunghezza 2 è detto trasposizione o scambio.

La scrittura ciclica di un l-ciclo non è unica. Se $l>1$, il ciclo ammette esattamente $l$ scritture cicliche distinte,
ottenute tramite rotazioni successive degli indici verso sinistra.

\section{NOTAZIONE}
Un ciclio è una lista di indici fra parentesi, e conveniamo che rappresenti la permutazione che associa a ogni indice nel ciclo
quello successivo.

\section{ESEMPIO}
Ad esempio, il ciclo
\[
 (12345)
\]
rappresenta la permutazione che manda 1 in 2, 2 in 3 e così via fino a 5 in 1. Due cicli sono disgiunti se non hanno lettere in comune.
Per esempio, (123) e (45) sono disgiunti, ma (123) e (124) no. 

\section{COMPOSIZIONE DI PERMUTAZIONI = PRODOTTO DI CICLI}
Per scrivere la composizione di permutazioni rappresentate da cicli,
basta scrivere i cicli di seguito.

Non è difficile calcolare la permutazione risultante da una composizione di cicli: basta, per ogni lettera, "seguire il suo destino" lungo
i vari cicli. Per esempio,
\[
 (123)(135)(24) = \left(\begin{array}{ccccc}
                         1 & 2 & 3 & 4 & 5 \\
                         4 & 5 & 3 & 2 & 1 Cicli\\
                        \end{array} \right)
\]
Come abbiamo fatto il conto? Cominciamo da 1: il primo ciclo manda 1 in 2, il secondo non tocca il 2, il terzo manda 2 in 4: concludiamo
che i tre cicli mandano 1 in 4. Il primo ciclo manda 2 in 3, il secondo 3 in 5, e il terzo non tocca 5: concludiamo che i tre cicli
mandano 2 in 5, e così via. Notate che alla fine del conto c'è un controllo di coerenza molto semplice:  tutti i numeri
nella seconda riga devono essere distinti.

\section{APPRFONDIMENTI}
\begin{itemize}
 \item DISPENSA: Orbite e cicli di una permutazione \cite{permutazione4}
 \item DISPENSA: Permutazioni \cite{permutazione2}
\end{itemize}



\section{DEFINIZIONE}
Decomporre una permutazione in cicli disgiunti vuol dire rappresentarla sotto forma di cicli.

\section{ESEMPIO}
Come fare a ottenere una rappresentazione in cicli di una permutazione? Basta "seguire" una lettera qualunque fino a trovare
un ciclo: per esempio, in 
\[
\left( \begin{array}{cccc} 1 & 2 & 3 & 4 \\ 3 & 1 & 2 & 4 \\ \end{array} \right) 
\]
abbiamo che 1 va in 3, 3 va in 2 e 2 va in 1; quindi il primo ciclo che troviamo è (123). A questo punto non ci rimane che 4,
che però va in sé, e formerebbe un ciclo di lunghezza 1. I cicli di lunghezza 1 per convenzione non si scrivono, e
quindi la permutazione si scrive (123).

NB: Secondo me se segui questo procedimento per forza di cose devi trovare cicli disgiunti.

\section{APPROFONDIMENTI}
\begin{itemize}
 \item DISPENSA: Permutazioni \cite{permutazione2}
 \item DISPENSA: Orbite e cicli di una permutazione \cite{permutazione4}
 \item DISPENSA: Lezione 9 \url{http://www.science.unitn.it/~luminati/didattica/md/1998/diario/Lezione_9.htm}
 \item ESERCIZI SVOLTI: Algebra 1 \url{http://www.mat.uniroma3.it/users/gabelli/AL1_06_07/soluzioni2esonero.pdf}
\end{itemize}



\section{DEFINIZIONE}
Data una qualsiasi permutazione, il suo periodo sarà il minimo comune multiplo dei periodi dei cicli disgiunti in cui essa si decompone. 

\section{NOTAZIONE}

\section{ESEMPIO}

\section{APPROFONDIMENTI}
\begin{itemize}
 \item FAQ \url{https://it.answers.yahoo.com/question/index?qid=20090214091406AAUSspi}
\end{itemize}




\section{INTRODUZIONE}
Ogni permutazione di $S_n$, $n>2$, è prodotto di trasposizioni. Osserviamo però che tali trasposizioni possono non essere
disgiunte ed inoltre la rappresentazione di una permutazione como prodotto di trasposizioni non è unica. Ad esempio, la permutazione
$\alpha=(123)$, si può scrivere come: $\alpha=(13)(12)=(12)(23)=(23)(13)$. Il teorema del segno di una permutazione ci dice
però che la parità (ovvero il segno) di una permutazione rimane la stessa.

\section{DEFINIZIONE}
Sia $\alpha \in S_n$, $n \geq 2$. Si dice che $\alpha$ è pari se è prodotto di un numero pari di trasposizioni, dispari se è prodotto di un 
numero dispari di trasposizioni.

Inoltre si dice che il segno di $\alpha$, $sgn(\alpha)$, è 1 se $\alpha$ è pari, -1 se $\alpha$ è dispari.

\section{NOTAZIONE}

\section{ESEMPIO}

\section{APPROFONDIMENTI}
\begin{itemize}
 \item \url{http://progettomatematica.dm.unibo.it/Permutazioni/fr6.htm}
\end{itemize}




\section{DEFINIZIONE}
L'ordine o periodo di un ciclo è uguale al numero di elementi del ciclo.

\section{NOTAZIONE}

\section{ESEMPIO}
Il ciclo (123) ha ordine 3.

\section{APPROFONDIMENTI}
\begin{itemize}
 \item FAQ \href{https://it.answers.yahoo.com/question/index?qid=20090214091406AAUSspi}{https://it.answers.yahoo.com/question/index?qid=20090214091406AAUSspi}
 \item TESI: Il gruppo simmetrico $S_n$ \cite{simmetrico1}
\end{itemize}




\section{DEFINIZIONE}
Un 2-ciclo si chiama anche scambio o trasposizione

\section{NOTAZIONE}

\section{ESEMPIO}
(12)

\section{APPROFONDIMENTI}
\begin{itemize}
 \item \url{http://web.tiscali.it/algebraastratta/mInsiemi/Permutazioni.htm}
\end{itemize}




\section{DEFINIZIONE}
Due permutazioni $\alpha$ e $\beta$ si definiscono disgiunte se gli oggetti che non sono fissi per una permutazione
sono fissi per l'altra, ovvere se:
\[
 (X \setminus F(\alpha))\cap (X \setminus F(\beta))=\o{}
\]


\section{NOTAZIONE}

\section{ESEMPIO 1}
Per esempio, (123) e (45) sono disgiunti, ma (123) e (124) no. 

\section{ESEMPIO 2}
In $S_{8}$, $\alpha = \left( \begin{array}{cccccccc} 1 & 2 & 3 & 4 & 5 & 6 & 7 & 8 \\ 3 & 2 & 4 & 7 & 5 & 6 & 1 & 8 \\ \end{array} \right)$
e $\beta = \left( \begin{array}{cccccccc} 1 & 2 & 3 & 4 & 5 & 6 & 7 & 8 \\ 1 & 8 & 3 & 4 & 5 & 6 & 2 & 8 \\ \end{array} \right)$
sono disgiunte, infatti $\{ 1,3,4,7 \} \cap \{ 2,8 \} = \o{}$

\section{APPROFONDIMENTI}
\begin{itemize}
 \item TESI DI LAUREA: Il gruppo simmetrico $S_{n}$ \cite{simmetrico1}
\end{itemize}




\section{DEFINITION}
A derangement is a permutation of the elements of a set, such that no element appears in its original 
position. \cite{derangement}

\section{NOTATION}
The number of derangement of a set of size $n$, usually written $D_{n}$, $d_{n}$, or $!n$, is called the "derangement number" or
"de Montmort number". (These numbers are generalized to rencontres numbers). \cite{derangement}

The number of derangements of an n-element set is called the nth derangement number or rencontres number, or the subfactorial
of n and is sometimes denoted $!n$ or $D_{n}$

\section{FORMULA DERANGEMENT}
\[
 d_{n} = n!\sum^{n}_{i=0} \frac{(-1)^i}{i!}
\]

\section{FORMULA PARTIAL DERANGEMENT}
La formula precendente è utilizzata quando vogliamo il numero delle permutazioni (o casi favorevoli, a volte negli esercizi) che hanno fixed point uguale a 0.
In generale per $k>0$ dove $k$ rappresenta il numero di fixed point, la formula diventa:

\[
 d_{n,k} = \frac{n!}{k!}\sum^{n}_{i=0} \frac{(-1)^i}{i!}
\]

\section{NOTE}
In altre parole, il derangment è un sottoinsieme dell'insieme delle permutazioni formato dalle permutazioni che non hanno punti fissi, cioè 
in cui nessun elemento è al suo posto.

\section{HISTORY}
The problem of counting derangements was first considered by Pierre Raymond de Montmort in 1708; he solved it in 1713, as did
Nicholas Bernoulli at about the same time. \cite{derangement}

\section{APPROFONDIMENTI}
\begin{itemize}
 \item \href{https://en.wikipedia.org/wiki/Derangement}{WIKIPEDIA: Derangement}
 \item \href{./pdf/DERANGEMENT/derangementLucchini.pdf}{DISPENSA: Derangement.pdf}
 \item \href{http://oeis.org/wiki/Number_of_derangements}{OEIS: Number of derangement}
\end{itemize}




\section{DEFINIZIONE}
Il principio di inclusione-esclusione è un'identità che mette in relazione la cardinalità di un insieme, espresso come unione di insiemi finiti,
con le cardinalità di instersezioni tra questi insiemi.

\section{HISTORY}
IL principio è stato utilizzato da Nicolaus II Bernoulli (1695-1726); la formula viene attribuita ad Abraham de Moivre (1667-1754);
per il suo utilizzo e per la comprensione della sua portata vengono ricordati Joseph Sylvester (1814-1897) ed Henri Poincaré (1854-1912). 

\section{APPROFONDIMENTI}
\begin{itemize}
 \item PAPER \cite{pinclescl}
\end{itemize}



