\chapter{Permutations}

\begin{definizione}
	Sia $X$ un insieme non vuoto. Si dice permutazione su $X$ ogni applicazione bigettiva di $X$ in se stesso.
\end{definizione}

Piano ragazzi, ci siamo gi\`a fottuti. Andiamo per gradi.

La prima cosa da fare \`e mettersi nell'ottica dell'idea che le formule combinatorie come la probabilit\`a hanno un modo per integrarsi nella matematica sotto forma di insiemi.

Giunti a questo punto dovremmo dire chi \`e la permutazione, il suo concetto e che strumentia abbiamo a disposizione per modellare tale concetto.

Sebbene abbia risvolti pratici nelle matematiche applicate, la permutazione \`e una questione di matematica pura e fondamenti della matematica. Per quest'ultime questioni possiamo affermare che la permutazione nasce con la nascita delle funzioni.

Ed infatti funzione e permutazione sono sinonimi, nel senso che una permutazione \`e una funzione, ed in particolare una funzione che associa ad un elemento di un insieme un altro elemento dello stesso insieme, possiamo dire che l'immagine dell'elemento prende il posto dell'elemento.

In generale, per indicare una permutazione si usano le lettere greche minuscole, es. $\sigma$, e la cosiddetta notazione matriciale,
nella quale sono riportarte (nella seconda riga) le immagini secondo $\sigma$ degli elementi di $X$ (scritti nella prima riga):
\[
\left(
\begin{array}{cccc}
1 & 2 & ... & n \\
\sigma(1) & \sigma(2) & ... & \sigma(n)
\end{array}
\right)
\] 

\begin{definizione}[Permutazione identica - elemento neutro rispetto alla composizione di permutazioni]
	In questa notazione, l'applicazione identica corrisponde ad una matrice con due righe uguali:
	\[
	\left(
	\begin{array}{cccc}
	1 & 2 & ... & n \\
	1 & 2 & ... & n \\
	\end{array}
	\right)
	\]
\end{definizione}

Indicheremo tale applicazione (detta permutazione identica), più semplicemente, con il simbolo $id$. 

\begin{definizione}[Inversa di una permutazione]
	per ottenere l’inversa di una permutazione basta scambiare la prima e la seconda riga e riordinare la prima. 
\end{definizione}

\begin{definizione}[Insieme delle permutazioni]
	Denoteremo con $S(X)$ l'insieme delle permutazioni su $X$. 
\end{definizione}

Il numero di elementi di $S(X)$ è uguale a $n!$, dove $n$ è il numero di elementi dell'insieme $X$.

\begin{definizione}[Ciclo di una permutazione (1)]
	Per \textit{ciclo di una permutazione} si intende il nome della notazione utilizzata per rappresentare una permutazione.
\end{definizione}

\begin{definizione}[Ciclo di una permutazione (2)]
	Sia $n$ un intero positivo. Si dice \textit{ciclo (o permutazione ciclica)} ogni $\sigma \in S_n$ per cui esistono un intero
	positivo $l$ e $a_1, ..., a_l \in \{1, ..., n\}$ a due a due distinti tali che
	\begin{itemize}
		\item $\sigma(a_1)=a_2, \sigma(a_2)=a_3, ..., \sigma(a_l)=a_1$;
		\item $\sigma(k)=k$ per ogni $k \in \{1, ..., n\} \setminus \{a_1, ..., a_l\}$.
	\end{itemize}
\end{definizione}

Il numero $l$ si dice lunghezza di $\sigma$. Una permutazione ciclica di lunghezza $l$ si dice anche l-ciclo.

\begin{definizione}[Ciclo di una permutazione (2)]
	Sia $r$ un intero positivo, $2 \leq r \leq n $ e siano dati $r$ elementi distinti $i_1, i_2, ..., i_r \in X=\{1,2,...,n\}$.
	Col simbolo $\gamma = (i_1 i_2 ... i_r)$ si denoti la permutazione $\gamma \in S_n$ tale che:
	\begin{enumerate}
		\item $\gamma (i_k) = i_k$ se $i_k \notin \{i_1, i_2, ..., i_r\}$
		\item $\gamma(i_k)=i_{k+1}$ se $1 \leq k \leq r-1$
		\item $\gamma (i_r)=i_1$
	\end{enumerate}
\end{definizione}
Tale permutazione è detta ciclo di lunghezza $r$.
Se il ciclo ha lunghezza 2 viene detto trasposizione o scambio.

Il solo ciclo di lunghezza 1 è la permutazione identica.

Il ciclo di lunghezza 2 è detto trasposizione o scambio.

La scrittura ciclica di un l-ciclo non è unica. Se $l>1$, il ciclo ammette esattamente $l$ scritture cicliche distinte,
ottenute tramite rotazioni successive degli indici verso sinistra.

Un ciclio è una lista di indici fra parentesi, e conveniamo che rappresenti la permutazione che associa a ogni indice nel ciclo
quello successivo.

\medskip
Ad esempio, il ciclo
\[
(12345)
\]
rappresenta la permutazione che manda 1 in 2, 2 in 3 e così via fino a 5 in 1. Due cicli sono disgiunti se non hanno lettere in comune.
Per esempio, (123) e (45) sono disgiunti, ma (123) e (124) no. 

\section{Composizione di permutazioni = Prodotto di cicli}
Per scrivere la composizione di permutazioni rappresentate da cicli,
basta scrivere i cicli di seguito.

Non è difficile calcolare la permutazione risultante da una composizione di cicli: basta, per ogni lettera, "seguire il suo destino" lungo
i vari cicli. Per esempio,
\[
(123)(135)(24) = \left(\begin{array}{ccccc}
1 & 2 & 3 & 4 & 5 \\
4 & 5 & 3 & 2 & 1 Cicli\\
\end{array} \right)
\]
Come abbiamo fatto il conto? Cominciamo da 1: il primo ciclo manda 1 in 2, il secondo non tocca il 2, il terzo manda 2 in 4: concludiamo
che i tre cicli mandano 1 in 4. Il primo ciclo manda 2 in 3, il secondo 3 in 5, e il terzo non tocca 5: concludiamo che i tre cicli
mandano 2 in 5, e così via. Notate che alla fine del conto c'è un controllo di coerenza molto semplice:  tutti i numeri
nella seconda riga devono essere distinti.

\begin{definizione}[Decomposizione di una permutazione in cicli]
	Decomporre una permutazione in cicli disgiunti vuol dire rappresentarla sotto forma di cicli.
\end{definizione}

Come fare a ottenere una rappresentazione in cicli di una permutazione? Basta "seguire" una lettera qualunque fino a trovare
un ciclo: per esempio, in 
\[
\left( \begin{array}{cccc} 1 & 2 & 3 & 4 \\ 3 & 1 & 2 & 4 \\ \end{array} \right) 
\]
abbiamo che 1 va in 3, 3 va in 2 e 2 va in 1; quindi il primo ciclo che troviamo è (123). A questo punto non ci rimane che 4,
che però va in sé, e formerebbe un ciclo di lunghezza 1. I cicli di lunghezza 1 per convenzione non si scrivono, e
quindi la permutazione si scrive (123).

NB: Secondo me se segui questo procedimento per forza di cose devi trovare cicli disgiunti.

\begin{definizione}[Periodo di una permutazione]
	Data una qualsiasi permutazione, il suo periodo sarà il minimo comune multiplo dei periodi dei cicli disgiunti in cui essa si decompone. 
\end{definizione}

INTRODUZIONE: \\
Ogni permutazione di $S_n$, $n>2$, è prodotto di trasposizioni. Osserviamo però che tali trasposizioni possono non essere
disgiunte ed inoltre la rappresentazione di una permutazione como prodotto di trasposizioni non è unica. Ad esempio, la permutazione
$\alpha=(123)$, si può scrivere come: $\alpha=(13)(12)=(12)(23)=(23)(13)$. Il teorema del segno di una permutazione ci dice
però che la parità (ovvero il segno) di una permutazione rimane la stessa.

\begin{definizione}
	Sia $\alpha \in S_n$, $n \geq 2$. Si dice che $\alpha$ è pari se è prodotto di un numero pari di trasposizioni, dispari se è prodotto di un 
	numero dispari di trasposizioni.
	
	Inoltre si dice che il segno di $\alpha$, $sgn(\alpha)$, è 1 se $\alpha$ è pari, -1 se $\alpha$ è dispari.
\end{definizione}

\begin{definizione}[Ordine o periodo di un ciclo di una permutazione]
	L'ordine o periodo di un ciclo è uguale al numero di elementi del ciclo.
\end{definizione}

ESEMPIO: \\
Il ciclo (123) ha ordine 3.



\begin{definizione}
	Un 2-ciclo si chiama anche scambio o trasposizione
\end{definizione}

ESEMPIO: \\
(12)

\begin{definizione}[Permutazioni disgiunte]
	Due permutazioni $\alpha$ e $\beta$ si definiscono disgiunte se gli oggetti che non sono fissi per una permutazione
	sono fissi per l'altra, ovvere se:
	\[
	(X \setminus F(\alpha))\cap (X \setminus F(\beta))=\o{}
	\]
\end{definizione}

ESEMPIO 1: \\
Per esempio, (123) e (45) sono disgiunti, ma (123) e (124) no. 

ESEMPIO 2: \\
In $S_{8}$, $\alpha = \left( \begin{array}{cccccccc} 1 & 2 & 3 & 4 & 5 & 6 & 7 & 8 \\ 3 & 2 & 4 & 7 & 5 & 6 & 1 & 8 \\ \end{array} \right)$
e $\beta = \left( \begin{array}{cccccccc} 1 & 2 & 3 & 4 & 5 & 6 & 7 & 8 \\ 1 & 8 & 3 & 4 & 5 & 6 & 2 & 8 \\ \end{array} \right)$
sono disgiunte, infatti $\{ 1,3,4,7 \} \cap \{ 2,8 \} = \o{}$


\begin{definizione}[Derangement]
	A derangement is a permutation of the elements of a set, such that no element appears in its original 
	position. 
\end{definizione}

\begin{definizione}[Number of derangement of a set]
	The number of derangement of a set of size $n$, usually written $D_{n}$, $d_{n}$, or $!n$, is called the "derangement number" or
	"de Montmort number". (These numbers are generalized to rencontres numbers). 
	
	The number of derangements of an n-element set is called the nth derangement number or rencontres number, or the subfactorial
	of n and is sometimes denoted $!n$ or $D_{n}$
\end{definizione}

\begin{definizione}[Formula Derangement]
	\[
	d_{n} = n!\sum^{n}_{i=0} \frac{(-1)^i}{i!}
	\]
\end{definizione}

\begin{definizione}[Formula partial derangement]
	La formula precendente è utilizzata quando vogliamo il numero delle permutazioni (o casi favorevoli, a volte negli esercizi) che hanno fixed point uguale a 0.
	In generale per $k>0$ dove $k$ rappresenta il numero di fixed point, la formula diventa:
	
	\[
	d_{n,k} = \frac{n!}{k!}\sum^{n}_{i=0} \frac{(-1)^i}{i!}
	\]
	
\end{definizione}

In altre parole, il derangment è un sottoinsieme dell'insieme delle permutazioni formato dalle permutazioni che non hanno punti fissi, cioè 
in cui nessun elemento è al suo posto.

The problem of counting derangements was first considered by Pierre Raymond de Montmort in 1708; he solved it in 1713, as did
Nicholas Bernoulli at about the same time. 

\begin{definizione}
	Il principio di inclusione-esclusione è un'identità che mette in relazione la cardinalità di un insieme, espresso come unione di insiemi finiti,
	con le cardinalità di instersezioni tra questi insiemi.
\end{definizione}

IL principio è stato utilizzato da Nicolaus II Bernoulli (1695-1726); la formula viene attribuita ad Abraham de Moivre (1667-1754);
per il suo utilizzo e per la comprensione della sua portata vengono ricordati Joseph Sylvester (1814-1897) ed Henri Poincaré (1854-1912). 



