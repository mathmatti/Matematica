\section{Fondamenti di analisi}

\subsection{Real Analysis through modern infinitesimal}
Real Analysis through Modern Infinitesimals (sostituire con bibitem) provide a course in mathematical analysis based on internal set theory (IST), introduced by Edward Nelson in 1977. After motivating IST through an ultrapower construction, the book provides a careful development of this theory, representing each external class as a proper class. This foundational discussion, which is presented in the first two chapters, includes an account of the basic internal and external properties of the real number system as an entity within IST. In its reamining 14 chapters, the book explores the perspective offered by IST as a wide range of real analysis topics are surveyed. The topics thus developed begin with those usually discussed in an advanced undergraduate analysis course and gradually move to those that are suitable for more advanced readers. This book may be used for reference, self-study, and as a source for advanced undergraduate or graduate courses.

This state of affairs can be ameliorated by upgrading the set-theoretical framework of mathematical analysis so as to allow a treatment of the subject that employs the \emph{method of modern infinitesimal}, or \emph{nonstandard analysis}. This upgrading will supplement mathematical analysis with methods that are remarkably simple in both conceptual and logica terms.

Readers who are interested in the philosofical issues concerning continuity and discreteness are invited to read an interesting book, \emph{The Continuous and the Infinitesimal in Mathematics and Philosophy}, by John L. Bell, Polimetrica 2005, which offers a thorough account of the historical development of these concepts.

The mathematical ideas needed for constructing such upgraded frameworks are manufactured in an important branch of mathematical logic called model theory. Model theory is essential for understanding how the basic properties of these frameworks derive from the manner in which they are constructed. Indeed (infatti), readers who are interested in the foundation of modern infinitesimals should study model theory.

But this book has a different aim. Our goal here is to explore the applications of modern infinitesimal in studying the central topics of real analysis.

Among the most rudimentary operations of the human mind are the acts of considering
a number of entities as a unit and 
of regarding a single object as a composed of a number of constituents 
(these are essentially the same mental operations that in other contexts are referred to as \emph{synthesizing} and \emph{analyzing}, respectively. \\
Ed infatti gli strumenti dell'analisi sono utilizzati per analizzare e sintetizzare. The intuitive notion of a collection of elements occurs to us in conjuction with these basic mental activities. 

An intellectual feat of the late nineteenth and early twentieth centuries was the discovery that virtually all know mathematics could be described in terms of ideas rooted in the intuitive notion of a "collation". This can be done once our intuitive understanding of collections and their properties acquires refinement, precision an suitable expansion through a linguistic model. The word \emph{set} is normally used in mathematics to refer to any such modified version of the intuitive notion of a collection. Indeed, such modifications have yielded sophisticated linguistic frameworks that serve as the foundations of mathematical and other scintific theories.

Each such foundational framework is called a \emph{set theory}. These is a core common to most set theories, known as \emph{elementary set theory}. The essential part of elementary set theory is concerned with the development of a language which, by being cast in symbols, endows with precision our intuitive understanding of the properties of collections. Most educated people today are expected to have some familiarity with elementary set theory.

More sophisticated theories of sets are obtained as we begin to expand our raw notion of a collection by assigning to it properties that may go beyond what is clearly possessed by physical or concrete examples of collections. But this transcendence entails controversies. One important controversy in the history of mathematics concerns the assumption that collections with infinitely many elements have actual (as opposed to potential) existence. One may argue, for example, that our ordinary experience of the physical world hardly informs us of the existence of such collections. Physicists tell us that even the whole universe consists of only finitely many particles.

So what makes us \emph{think} of infinite sets? We may answer this question in one word - a continuum. We perceive the physical world in terms of time and space, which the mind grasps as continua. Herein lies the origin of geometric concepts such as straight lines, curves, planes, and surfaces. Such geometric concepts have historically been called consinuous quantities. One fruitful method of studying consinuous quantities is through the analytic models that we construct for them. The notion of an infinite set in its na\"ive form naturally occurs to us as we attempt to construct such models.

But it is one thing to be in possession of a raw idea, and quite another to define an impeccable mathematical concept that can blossom into a sophisticated theory. Indeed, in its na\"ive form the notion of an infinite set has been around since ancient times. But we were not in possession of a rigorous mathematical theory of infinite sets until the turn of the twentieth century. And, when this happened, the result came to be regarded as a watershed in the progress of rational thought. Such is the status of infitite sets in the mathematics of our time.

\section*{Funzioni elementari}
\begin{itemize}
 \item \href{Tangent.pdf}{Tangent}
 \item \href{Arcsine.pdf}{Arcsine}
 \item Valore assoluto, esponenziali, logaritmi, radici, equazioni e disequazioni
 \item \href{./ValoreAssoluto.pdf}{Valore assoluto}
 \item \href{./FunzioneEsponenziale.pdf}{Funzione esponenziale} 
\end{itemize}

\section{Analisi I}
\begin{itemize}
 \item \href{FunzioneContinua.pdf}{Funzione continua}
 \item \href{Limite.pdf}{Limite}
 \item \href{Weierstrass.pdf}{TEOREMA di Weierstrass. Una funzione continua in un insieme $E$ compatto ha massimo e minimo.}
 \item Successione
 \item \href{Derivata.pdf}{Derivata} 
 \item Integrale
 \item Lebensque
\end{itemize}

\section{Analisi II}
\begin{itemize}
 \item Serie in $R$ e $C$. 
 \item Successioni e serie di funzioni: convergenza puntuale e uniforme. 
 \item Serie di potenze. 
 \item Sviluppabilità in serie di Taylor. 
 \item Calcolo differenziale per funzioni di più variabili reali. 
 \item Teoremi del valor medio. 
 \item Formula di Taylor. 
 \item Funzioni convesse. 
 \item Massimi e minimi locali. 
 \item Invertibilità locale, funzioni implicite. 
 \item Estremi vincolati (moltiplicatori di Lagrange). 
 \item Il teorema del punto fisso per le contrazioni.
 \item Equazioni e sistemi di equazioni differenziali ordinarie, problema di Cauchy: esistenza locale e prolungabiltà delle soluzioni; metodi risolutivi per equazioni di tipo particolare. 
 \item Equazioni e sistemi lineari: integrale generale, risoluzione di equazioni e sistemi a coefficienti costanti. 
 \item Elementi di teoria della misura e integrazione secondo Lebesgue in $R^N$. 
 \item Cenni su integrali curvilinei, campi vettoriali, potenziale.
\end{itemize}

\section{Analisi superiore}
\begin{itemize}
 \item Elementi di teoria astratta della misura. 
 \item Misure di Borel e di Radon. 
 \item Decomposizione di Lebesgue, Teorema di Radon-Nikodym.
 \item Spazi $L^p$: completezza, densita' di alcune classi di funzioni (funzioni semplici, funzioni continue), regolarizzazione (mollificatori di Friedrichs). 
 \item Spazi di Banach e spazi di Hilbert. 
 \item Teorema di Baire. 
 \item Teoremi di Hahn-Banach, di Banach-Steinhaus e del grafico chiuso. 
 \item Trasformata di Fourier in $L^1$, nello spazio di Scwarz S e in $L^2$. 
 \item Derivate deboli e spazi di Sobolev. 
 \item Teorema di Lax-Milgram e problema di Dirichlet per operatori ellittici del secondo ordine.
\end{itemize}



\section{Funzione}
La parola "funzione" appare per la prima volta verso la fine del XVII secolo, nella corrispondenza tra Leibniz e Johann Bernoulli; tuttavia \'{e} solo con l'opera di Eulero
che questo concetto si afferma come uno dei principali strumenti dell'analisi.
