\section*{Funzioni elementari}
\begin{itemize}
 \item \href{Tangent.pdf}{Tangent}
 \item \href{Arcsine.pdf}{Arcsine}
 \item Valore assoluto, esponenziali, logaritmi, radici, equazioni e disequazioni
 \item \href{./ValoreAssoluto.pdf}{Valore assoluto}
 \item \href{./FunzioneEsponenziale.pdf}{Funzione esponenziale} 
\end{itemize}

\section{Analisi I}
\begin{itemize}
 \item \href{FunzioneContinua.pdf}{Funzione continua}
 \item \href{Limite.pdf}{Limite}
 \item \href{Weierstrass.pdf}{TEOREMA di Weierstrass. Una funzione continua in un insieme $E$ compatto ha massimo e minimo.}
 \item Successione
 \item \href{Derivata.pdf}{Derivata} 
 \item Integrale
 \item Lebensque
\end{itemize}

\section{Analisi II}
\begin{itemize}
 \item Serie in $R$ e $C$. 
 \item Successioni e serie di funzioni: convergenza puntuale e uniforme. 
 \item Serie di potenze. 
 \item Sviluppabilità in serie di Taylor. 
 \item Calcolo differenziale per funzioni di più variabili reali. 
 \item Teoremi del valor medio. 
 \item Formula di Taylor. 
 \item Funzioni convesse. 
 \item Massimi e minimi locali. 
 \item Invertibilità locale, funzioni implicite. 
 \item Estremi vincolati (moltiplicatori di Lagrange). 
 \item Il teorema del punto fisso per le contrazioni.
 \item Equazioni e sistemi di equazioni differenziali ordinarie, problema di Cauchy: esistenza locale e prolungabiltà delle soluzioni; metodi risolutivi per equazioni di tipo particolare. 
 \item Equazioni e sistemi lineari: integrale generale, risoluzione di equazioni e sistemi a coefficienti costanti. 
 \item Elementi di teoria della misura e integrazione secondo Lebesgue in $R^N$. 
 \item Cenni su integrali curvilinei, campi vettoriali, potenziale.
\end{itemize}

\section{Analisi superiore}
\begin{itemize}
 \item Elementi di teoria astratta della misura. 
 \item Misure di Borel e di Radon. 
 \item Decomposizione di Lebesgue, Teorema di Radon-Nikodym.
 \item Spazi $L^p$: completezza, densita' di alcune classi di funzioni (funzioni semplici, funzioni continue), regolarizzazione (mollificatori di Friedrichs). 
 \item Spazi di Banach e spazi di Hilbert. 
 \item Teorema di Baire. 
 \item Teoremi di Hahn-Banach, di Banach-Steinhaus e del grafico chiuso. 
 \item Trasformata di Fourier in $L^1$, nello spazio di Scwarz S e in $L^2$. 
 \item Derivate deboli e spazi di Sobolev. 
 \item Teorema di Lax-Milgram e problema di Dirichlet per operatori ellittici del secondo ordine.
\end{itemize}



\section{Funzione}
La parola "funzione" appare per la prima volta verso la fine del XVII secolo, nella corrispondenza tra Leibniz e Johann Bernoulli; tuttavia \'{e} solo con l'opera di Eulero
che questo concetto si afferma come uno dei principali strumenti dell'analisi.
