\chapter{How to study Game Theory}
La teoria dei giochi deve essere studiata da due angolazioni, da due facce della stessa medaglia potremmo dire forzando un p\`o la fantasia. Ossia, il punto di vista della realt\`a che si vuole modellare, quindi per esempio l'economia ed il punto di vista del modello matematico sottostante. \`E chiaro che i due punti di vista hanno approcci e metodi differenti ma non \`e solo questo il punto. La dicotomia si concretizza nel fatto che l'insieme dei player sia rappresentato, per esempio, dal player set: $\playerSet$. Cio\`e una volta stabilita la corrispondenza uno a uno, biunivoca tra realt\`a e matematica, possiamo tralasciare il punto di vista della real\`a ovvero possiamo disinteressarcene e prendiamo a considerare soltanto l'oggetto matematico rappresentato in questo caso dal player set $\playerSet$. Se considero l'insieme power set $\powerSet$, che cosa sto facendo? Dal punto di vista della matematica una cosa assolutamente lecita, devo capire di cosa si tratta e come faccio a manipolarla e dal punto di vista del ritorno alla realt\`a posso dire in questo caso che $\powerSet$ rappresenta l'insieme di tutte le possibili coalizioni di giocatori. Il power set lo vedremo e rivedremo quindi niente paura.

Ed infatti i padri della disciplina parlano di shift (spostamento) da teoria economica a matematica (e viceversa), cio\`e lo spostamento da realt\`a a modello matematico.

\section{Shift of Emphasis from Economics to Games. \cite{vonNeumann1944}}
Chapter II, GENERAL FORMAL DESCRIPTION OF GAMES OF STRATEGY, 5. Introduzione, 5.1 Shift of Emphasis from Economics to Games, \cite{vonNeumann1944}.

\quote{It should be clear from the discussions of Chapter I that a theory of rational behaviour - i.e. of the foundations of economics and of the main mechanisms of social organization - requires a thorough study of the "games of strategy." Consequently we must now take up the theory of games as an independent subject. In studying it as a problem in its own right, our point of view must of necessaty undergo a serious shift. In Chapter I our primary interest lay in economics. It was after having concived ourselves of the impossibility of making progress in that field without a previous fundamental understangin of the games that we gradually approached the formulations and the questions which are partial to that subject. But economic viewpoints remained nevertheless the dominant ones in all of Chapter I. From this Chapter II on , however, we shall have to treat the games as games. Therefore we shall not mind if some points taken up have no economic connections whatever, - it would not be possible to do full justice to the subject otherwise. Of course most of the main concepts are still those familiar from the discussions of economic literature (cf. the next section) but the deails will often be altogether alien to it - and details, as usual, may dominate the exposition and overshadow the guiding principles.}
