\chapter{Strategies v. 10/10/2017}
\begin{itemize}
	\item In simultaneous-move games all player move or take action simultaneously, hence choosing a strategy is the same as choosing an action. This is no longer true in multistage games, where choosing a strategy means choosing a sequence of (conditional) actions. Although in this course non-cooperative games shall be dealt with only in simultaneous-move form, still multistage games are briefly introduced hereafter to formalize both a general definition of strategies and the notion of (in)complete information.
	\item I - baudo - Multistage games vengono introdotti per generalit\`a e per modellare la nozione di (in)complete information.
	\item before describing multistage games, the simultaneous-move setting enables to distinguish beetween outcomes of the game and strategy/action profiles. To this end, let each player $i \in N$ choose an action from a finite set $\actionSet$, where $|A_i| \ge 2$ for all $i \in N$. The product space $\actionProfileSet$ contains all $n$-tuples or profiles of actions, with generic element $a = (a_1, ..., a_n) \in A$. Preferences  
\end{itemize}


\subsection{Multistage games}
\subsection{Dominated and dominant strategies}
\subsubsection{Strategy deletion}
\subsubsection{Prisoner's dilemma}

\section{Randomization and expected payoffs}
\subsection{Mixed strategies}
\subsection{Domination in mixed strategies}
\subsection{Best responses}
\subsection{Nash equilibrium}
\subsection{Strong equilibrium}