\documentclass[a4paper,10pt]{article}
\usepackage[utf8]{inputenc}
\usepackage{amsmath}
\usepackage{hyperref}

%opening
\title{Esercizio 4: Applicazione del principio di inclusione-esclusione; numero di permutazioni senza punti fissi (derangement); }
\author{baudo81[at]gmail.com}

\begin{document}

\maketitle



\section{TESTO}
Un postino sbadato consegna a caso 6 raccomandate a 6 destinatari.
Qual'è la probabilità che:
\begin{itemize}
 \item almeno uno riceva la propria;
 \item esattamente 2 non ricevano ricevano la propria? (NOTA: Equivale a dire il contrario: esattamente 4 ricevano la propria. Per risolvere questo punto dovrebbe essere necessario leggere \href{./Derangement.html}{Derangement, partial derangement}
\end{itemize}


\section{TEORIA}
\begin{itemize}
 \item \href{./Permutazioni.html}{Permutazioni}
 \item \href{./Derangement.html}{Numero di permutazioni senza punti fissi (Derangement)}
 \item \href{./PrincipioInclusioneEsclusione.html}{Principio di inclusione-esclusione}
\end{itemize}


\section{SOLUZIONE}
\begin{itemize}
 \item Il numero di casi favorevoli è dato dal numero di casi totali, $6!$, da cui dobbiamo togliere il numero di casi in cui nessuno
 riceve la propria raccomandata, cioè il numero di permutazioni di 6 elementi senza punti fissi, 
 \[d_{6} = 6!\sum^{6}_{i=0} \frac{(-1)^i}{i!} \]
 
 Ora dobbiamo dividere questo numero per il numero di casi totali, ottenendo:
 \[
  \frac{6! - 6!\sum^{6}_{i=0} \frac{(-1)^i}{i!}}{6!}
 \]

 
 \item Per trovare il numero di casi favorevoli dobbiamo solo scegliere i due che non ricevono la loro raccomandata. Questo può essere
 fatto in ${6 \choose 2}$.
 Ora dividiamo per il numero di casi totali,
 \[
  \frac{{6 \choose 2}}{6!}
 \]

\end{itemize}

\end{document}
