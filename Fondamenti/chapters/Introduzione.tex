\chapter{Introduzione ai fondamenti della Matematica}

Vi sarete di certo chiesti: \textit{ho letto questa dimostrazione per la decima volta e ancora non ci capisco nulla, perch\'e?}, \'e lecito chidersi perch\'e dopo tutto la dimostrazione di qualsivoglia teorema non la devo mica inventare, la vorrei
semplicemente leggere come in un libro di cucina. In questo viene in aiuto lo studio dei fondamenti della matematica, chiamata da Hilber metamatematica.
 
Metamathematics is the study of the tools or techniques needed by an undergraduate student in order to be able to study mathematics profitably. 

E, di fatto, Hilbert, essendo un matematico, la sa lunga sull'uso delle parole per definire gli oggetti. La metamatematica è lo studio della matematica con l'utilizzo della stessa matematica. Probabilmente il termine moderno di metamatematica \'e \textit{logica matematica}.

Bene, una volta che abbiamo imparato che cos'\'e la metamatematica ovvero abbiamo acquisito una serie di strumenti della logica matematica possiamo iniziare a studiare matematica, e perch\'e non iniziare dalla teoria degli insiemi.

La teoria ingenua degli insiemi è un modello matematico che si pone alla base di tutta la matematica.  
Putroppo successivamente si scoprirono dei bug ovvero
il modello portava a dei paradossi e quindi fu aggiustata da Zermelo, Fraenkel e Skolem (ZFS). 

Ecco il modello ingenuo:  
Un insieme è una collezione di elementi distinti con due particolarità:  
\begin{itemize}
 \item gli elementi dell'insieme possono essere, a loro volta, insiemi.
 \item esiste un insieme composto da nessun insieme, detto insieme vuoto.
\end{itemize}




\section{Derivata}
Il calcolo differenziale nasce soprattutto per affrontare problemi di geometria, in particolare quello delle tangenti.




\chapter*{PREFAZIONE}
Il lavoro che segue non \'{e} la ristampa di un trattato precedente dello stesso autore, intitolato \textit{L'analisi matematica della logica}. Le sue prime parti sono, \'{e} vero,
dedicate allo stesso argomento, e il libro incomincia stabilendo il medesimo sistema di leggi fondamentali, ma i suoi metodi sono pi\'{u} generali e il campo delle sue applicazioni
\'{e} di gran lunga pi\'{u} ampio. Il libro espone i risultati, maturati in anni di studio e di riflessione, di un principio d'indagine relativo alle operazioni dell'intelletto, la cui
prima esposizione fu stesa in poche settimane da che fu concepita l'idea.

\textbf{Questo e il secondo lavoro di boole ma e da considerare quello principale. Per fare un analogia struts... Bisogna prima di leggere questo trattato
  leggere altri due trattati che erano le basi per poter leggere il libro di Boole nel 1850.  
  \textit{Elementi di logica} dell'arcivescovo Whately oppure \textit{Lineamenti delle leggi del pensiero} del signor Thomson.
}

\chapter{Natura e scopo dell'opera}

\section{}
Scopo di questo trattato \'{e} d'indagare le leggi fondamentali di quelle operazioni della mente per mezzo delle quali si attua il ragionamento; di dar loro espressione nel linguaggio
simbolico di un calcolo e d'istituire, su questo fondamento, la scienza della logica costruendone il metodo; di fare, di questo stesso metodo, la base di un metodo generale per
l'applicazione della dottrina matematica della probabilit\'{a} e, in ultimo, di ricavare dai diversi elementi di verit\'{a} portati alla luce nel corso di queste indagini alcune
indicazioni probabili sulla natura e la costituzione della mente umana.

\section{}
Non c'\'{e} quasi bisogno di ricordare che questo progetto non \'{e} del tutto originale, e tutti sanno che i filosofi hanno dedicato una parte considerevole della loro attenzione
a quelle che sono, dal punto di vista pratico, le sue suddivisioni principali: la logica e la teoria della probabilit\'{a}. Nella forma che le fu data dagli antichi e dagli Scolastici,
la logica \'{e} quasi esclusivamente associata al grande nome di Aristotele: e fino ai giorni nostri, salvo alcuni cambiamenti inessenziali, \'{e} rimasta praticamente tal quale fu
presentata all'antica Grecia nelle disquisizioni in parte tecniche e in parte metafisiche dell'\textit{Organon}. Dal canto suo, l'indirizzo della ricerca originale si \'{e} orientato
principalmente verso questioni di filosofia generale le quali, pur essendo sorte fra le dispute dei logici, sono andate oltre quello che erano all'origine, dando alle epoche successive
della speculazione la loro inclinazione e il loro carattere particolari. Le et\'{a} di Porfirio e Proclo, di Anselmo e Abelardo, di Ramus e Descartes, conclusesi con la contestazione
di Bacone e Locke, stanno davanti alla nostra mente come esempi delle pi\'{u} remote influenze che questo studio ha esercitato sul cammino del pensiero umano: in parte perch\'{e} hanno
suggerito fecondi argomenti di discussione, in parte perch\'{e} hanno dato luogo alle critiche contro le sue pretese illegittime. Dall'altra parte, la storia della teoria della
probabilit\'{a} \'{e} stata contraddistinta in misura molto maggiore da quel costante sviluppo che costituisce la caratteristica propria della scienza. Il genio precoce di Pascal alle origini
di questa disciplina, le pi\'{u} profonde tra le speculazioni matematiche di Laplace nelle sue fasi pi\'{u} mature (e qui non faccio menzione di altri nomi, non meno noti di questi)
furono impegnati nel perfezionamento di questa teoria. Como lo studio della logica ha esercitato la propria influenza sul pensiero per le questioni di metafisica, ad esso affini,
cui ha dato occasione, cos\'{i} quello della teoria della probabilit\'{a} deve ritenersi importante per lo sviluppo che ha impresso alle parti pi\'{u} astratte della scienza matematica.
Si \'{e} inoltre ritenuto giustamente che ciascuna di queste discipline avesse di mira, oltre che fini pratici, anche fini teorici. L'oggetto della logica, infatti, non \'{e} solo
quello di metterci in grado di trarre inferenze corrette da premesse date, n\'{e} l'unica pretesa della teoria della probabilit\'{a} \'{e} quella di insegnarci come fondare su solide
basi il mestiere di assicuratore sulla vita o di raccogliere in formule i dati significativi delle innumerevoli osservazioni che si compiono in astronomia, in fisica, o in quel campo delle ricerche
sociali che oggi va rapidamente acquistando importanza. Entrambi questi studi presentano anche un interesse di altro genere, derivante dalla luce che gettano sui poteri dell'intelletto.
C'insegnano in qual modo il linguaggio e il numero servano da strumento e da ausilio ai processi del ragionamento; ci rivelano, in certa misura, la connessione esistente fra i diversi
poteri del nostro comune intelletto; mettono davanti a noi, nei due domin\^{i} della conoscenza dimostrativa e di quella probabile, i modelli essenziali della verit\'{a} e della
correttezza: modelli che non sono stati ricavati dall'esterno, ma sono profondamente radicati nella costituzione delle facolt\'{a} dell'uomo. Questi fini speculativi non cedono n\'{e}
in dignit\'{a}, n\'{e}, si potrebbe aggiungere, in importanza, agli scopi pratici con il perseguimento dei quali sono stati spesso associati nel corso della loro storia. Lo svelare le
leggi e le relazioni pi\'{u} nascoste di quelle facolt\'{a} superiori del pensiero grazie alle quali giungiamo a possedere, o portiamo a compimento, tutto ci\'{o} che va oltre la pura
e semplice conoscenza percettiva del mondo e di noi stessi, \'{e} un fine la cui dignit\'{a} non ha certo bisogno di essere raccomandata a uno spirito raziocinante.

\chapter{Dei segni in generale e dei segni adatti alla scienza della logica in particolare; delle leggi alle quali \'{e} sottoposta quest'ultima classe di segni}



\section{DEFINIZIONE}
Aristotelian or traditional logic is a “subject-predicate” logic and is therefore concerned only with a portion of the sum total of logical truth.

It confines itself to the four forms of categorical proposition 
known as the A, E, I, and O forms.

In the second place, it treats subalternation as a valid form of inference. 
That is, it assigns (tacitly at least) conventional meanings, different from those employed in modern 
(i.e. symbolic or mathematical) logic, to the four categorical forms, meanings such that subalternation holds.

These two characteristics, however, are not sufficient to define Aristotelian logic. 
For an indefinite number of systems could be devised which possess these two properties. 
Since further distinguishing traits of Aristotelian logic will appear only in the course of our investigation, 
a definition of Aristotelian logic framed in terms of its essential properties cannot be given at the outset.


\section{DEFINIZIONE}
Constructive mathematics is distinguished from its traditional counterpart, classical mathematics, 
by the strict interpretation of the phrase “there exists” as “we can construct”. 
In order to work constructively, we need to re-interpret not only the existential quantifier but all the logical connectives and quantifiers as instructions on how to construct 
a proof of the statement involving these logical expressions. 


\section{DEFINIZIONE}
Intuitionistic logic, sometimes more generally called constructive logic, refers to systems of symbolic logic that differ from the systems used for classical logic 
by more closely mirroring the notion of constructive proof. 
In particular, systems of intuitionistic logic do not include the law of the excluded middle and double negation elimination, 
which are fundamental inference rules in classical logic. \cite{k1}

Intuitionistic logic encompasses the principles of logical reasoning which were used by L. E. J. Brouwer in developing his intuitionistic mathematics, 
beginning in [1907]. Because these principles also underly Russian recursive analysis and the constructive analysis of E. Bishop and his followers, 
intuitionistic logic may be considered the logical basis of constructive mathematics. 



\section{Elementi di logica dell'arcivescovo Whately}


\subsection{DEFINIZIONE}
Gli \textit{Elementi di logica} del 1853 dell'arcivescovo Whately era il testo che Boole consiglia di leggere prima di leggere il suo trattato del 1954.


\section{DEFINIZIONE}
Dato un insieme numerabile di simboli \'{e} possibile trovare, per ogni formula, un algoritmo che mi dica se quella formula \'{e} o no derivabile dall'insieme di simboli e dalle regole date.


\section{DEFINIZIONE (1)}
An interpretation is an assignment of meaning to the symbols of a formal language.  

assignment = ???  
symbol = ???  
formal language = ???

\section{DEFINIZIONE (2)}
An interpretation $I$ is a function which assigns to any atomic formula $p_i$ a truth value

\[
I(p_i) \in {0,1}
\]

If $I(p_i)=1$ then $p_i$ is called true under the interpretation $I$  
If $I(p_i)=0$ then $p_i$ is called false under the interpretation $I$

\section{DEFINIZIONE (2)}
An interpretation $A$ is a structure





\section{Introduzione - Axiom of extensionality}
Dovrebbe essere abbastanza evidente che, l'uguaglianza tra due strutture matematiche, \'{e} un fatto "quasi" arbitrario e quindi in un certo senso devo definire che cosa si intende
per uguaglianza tra due insiemi. Ebbene, Zermelo identifica tale concetto di uguaglianza con il concetto di estenzionalit\'{a}, cio\'{e} di estensione. Dietro questo assiome
c\'{e} dell'altro. In un certo senso, si potrebbe dire che \'{e} il meccanismo del contare all'interno della teoria degli insiemi. Quindi dall'assioma non ricavo "quanti" sono
gli elementi dei due insiemi ma sono certo del fatto che i due insiemi contengono lo "stesso numero" di elementi.

\section{NOTAZIONE}
\[
  \forall A \forall B (\forall X (X \in A \Leftrightarrow X \in B ) \Rightarrow A = B)
\]




\section{Passaggio da linguaggio naturale a linguaggio simbolico (linguaggio formale?)}
\subsection{ESEMPI}
%If $x$ is an ancestor of $y$ is an ancestor of $z$, then $x$ is an ancestor of $z$. $P(x,y) \and P(y,z) \to P(x,z)$



\section{Logica e Fondamenti}


\section{Introduction}
La logica si divide in logica dei predicati e logica delle proposizioni. La logica dei predicati è la logica delle proposizioni con l'aggiunta dei quantificatori
per ogni ($\forall$) ed esiste ($\exists$). Quando si passa alla logica dei predicati sembra che le proposizioni vengano rappresentate come delle funzioni. 

\section{Syllabus}
\begin{itemize}
 \item Notion
 \item Symbol
 \item Variable (symbol)
 \item Language
 \item Reasoning
 \item Magnitude
 \item Law of excluded middle
 \item Propositional function
 \item \href{Appartenenza.html}{Appartenenza}
 \item Inclusione
 \item Uguaglianza (tra insiemi)
 \item \href{TeoriaIngenuaInsiemi.html}{Teoria ingenua degli insiemi}
 \item Cantor = teoria dei numeri e teoria (ingenua) degli insiemi
 \item \href{./Esistenza.html}{Esistenza}
 \item \href{./Transfinito.html}{Transfinito}
 \item Coerenza
 \item Indipendenza
 \item \href{./SviluppiDecimali.html}{Sviluppi decimali}
 \item Funziona caratteristica di un sottoinsieme.
 \item Rappresentazione posizionale dei numeri
 \item \href{./Interpretation.html}{Interpretation}
 \item \href{./Passaggio.html}{Passaggio da linguaggio naturale a linguaggio simbolico (linguaggio formale?)}
 \item \href{ClassicalLogic.html}{Classical Logic}
 \item \href{FormalLogic.html}{Formal Logic}
 \item \href{./IntuitiveLogic.html}{Intuitive Logic}
 \item Constructive Logic
 \item \href{ConstructiveMathematics.html}{Constructive Mathematics}
 \item Symbolic Logic
 \item \href{Entscheidungsproblem.html}{Entscheidungsproblem}
 \item \href{./RecursiveFunctionTheory.html}{Recursive Function Theory}
\end{itemize}



\section{Temi d'esame}
  \begin{itemize}
   \item 1
  \end{itemize}

\section{Dispense di partenza}
\begin{itemize}
  \item Dispense Prof. Placci Unibo
  \item \url{http://www.settheory.net/}
  \item \url{http://www.mafy.lut.fi/study/LogicAndDiscreteMethods/Lectures/Lecture2.pdf}
  \item \url{http://cse.unl.edu/~choueiry/F07-235/files/PredicatesQuantifiers.pdf}
  \item \url{http://cgi.csc.liv.ac.uk/~frank/teaching/comp118/lecture2.pdf}
  \item \url{http://emilkirkegaard.dk/en/wp-content/uploads/0415400678.Routledge.Logic_.An_.Introduction.Dec_.2005.pdf}
\end{itemize}

\section{GOOGLE SEARCHES}
\begin{itemize}
 \item predicate logic solved examples
\end{itemize}

\section{Altre pagine latex da integrare in questi appunti}
\begin{itemize}
 \item \url{https://github.com/baudo2048/appunti}
\end{itemize}




