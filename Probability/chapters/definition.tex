\chapter{Definition of Probability}

\section{Numero aleatorio}
bla

\section{Probabilities as set functions. \cite{rossi26102017}}
\begin{itemize}
 \item Conceptually, probabilities are associated with \emph{events or subsets} $A \in 2^X$ of atomic mutually exclusive events $x \in X$. 
       In fact, a probability distribution is a \emph{set function} satisfying $p(A)+p(B)=p(A \cap B) + p(A \cup B)$ for all $A,B \in 2^X$.
\end{itemize}

Quindi si richiede che gli elementi dell'insieme X dei possibili eventi (qui c\`e una certa ridondanza), piuttosto utilizzerei la terminilogia adottata da:

Intanto che cosa \`e un evento? Un evento \`e un numero aleatorio ossia un numero rappresentato da una lista non ordinata di valori.

Es. $x_1 = \{1,2,3,4,5\}$ oppure $x_2 = \{0,1\}$ oppure $x_3 = \{7\}$ $x_1$, $x_2$ e $x_3$ vengono chiamati numeri aleatori. $x_3$ pur essendo un insieme si potrebbe far corrispondere al numero $7$. $x_2$ si chiama evento perch\`e pu\`o assumere soltanto due valori $0$ e $1$.


