\documentclass[a4paper,10pt]{article}
\usepackage[utf8]{inputenc}
\usepackage{amsmath}
\usepackage{hyperref}

%opening
\title{Arcsine}
\author{\href{http://www.baudo.hol.es}{giuseppe baudo}}

\begin{document}

\maketitle

\section{DEFINIZIONE}

\section{NOTAZIONE}
$\arcsin(x)$

\section{ESEMPIO}

\section{NOTE}
Sappiamo che il dominio della funzione $\sin(x)$ \'{e} tutto $R$ mentre il codominio (range) \'{e} l'intervallo $[-1,1]$. La funzione $\sin(x)$ non ha dunque inversa?
Allora ecco perchè si considera un suo restringimento ossia si considera come dominio solo l'intervallo $[-\frac{\pi}{2},\frac{\pi}{2}]$. Ricordati di invertire gli insiemi
quando pensi all'arcoseno e quindi all'inversa.

\section{APPROFONDIMENTI}
\begin{itemize}
 \item \url{https://en.wikipedia.org/wiki/Inverse_trigonometric_functions}
 \item \url{http://www.rapidtables.com/math/trigonometry/arcsin.htm}
 \item \url{https://www.youtube.com/watch?v=ZX1EiSoVenU}
\end{itemize}

\end{document}
